\documentclass[12pt,a4paper]{article}
    % for Nature's style citations
\usepackage[super]{natbib}
\usepackage[utf8x]{inputenc}
\usepackage[greek,francais]{babel}
\usepackage[T1]{fontenc}
\usepackage{graphicx}
\usepackage{fancyhdr} % Needed to define custom headers/footers
\usepackage{setspace}
\usepackage{listings}
\usepackage{color}
\usepackage[scale=2]{ccicons}
\usepackage{caption}
\usepackage{url}
\usepackage[final]{pdfpages}
\usepackage{caption}
\usepackage{svg}
\usepackage{listings}

\usepackage{eurosym}
\usepackage{wrapfig}
\usepackage{subfig}
\usepackage[margin=2.7cm]{geometry}

\usepackage{amsmath}
\usepackage{caption}
 
\DeclareCaptionType{mycapequ}[][List of equations]
\captionsetup[mycapequ]{labelformat=empty}
 
    

    %%%%%%%%%%%%%%%%%%%%%%%%
    %%%%% MISE EN PAGE %%%%%
    %%%%%%%%%%%%%%%%%%%%%%%%
    %Interligne à 1.5
    \onehalfspacing





    \pagestyle{fancy} % Enables the custom headers/footers
    \lhead{Thèse de médecine }
    \chead{UBO}
    \rhead{Sacha SCHUTZ}
    \lfoot{truc}
    \rfoot{2015/2016}
     
     
    %%%%%%%%%%
    % MACROS %
    %%%%%%%%%%
    \newcommand{\HRule}{\rule{\linewidth}{0.5mm}} % Defines a new command for the horizontal lines, change thickness here
    \newcommand\nt{nucléotides }
     
    \newcommand{\includefigure}[3] {% label, caption, width ratio
    %ex\includefigure{RkNN}{Graphe d'exemple et R1NN/R2NN associés.}{0.8}
            \begin{center}
         \includegraphics[width=#3\textwidth]{#1}
         \captionof{figure}{#2}
         \label{FIG:#1}
            \end{center}
    }
     

\renewcommand{\thefootnote}{\roman{footnote}}
\begin{document}


%%%%%%%%%%%%%%
% TITLE PAGE %
%%%%%%%%%%%%%%
\begin{titlepage}
\center







%       HEADING SECTIONS
\textsc{\LARGE Thèse de médecine}\\[1.5cm]
\textsc{\Large CHU-BREST}\\[0.5cm]
\textsc{\large Université de Brest\\
DES de biologie médicale \
}\\[0.5cm]
 
%       TITLE SECTION
\HRule \\[0.8cm]

{ \huge \bfseries Description du microbiote pulmonaire chez les patients atteints de mucoviscidoses}\\[0.4cm]

\HRule \\[1.2cm]
 
%       AUTHOR SECTION
\begin{minipage}{0.4\textwidth}
 \begin{flushleft} \large
     \emph{Auteur:}\\
     Sacha SCHUTZ
 \end{flushleft}
\end{minipage}
~
\begin{minipage}{0.4\textwidth}
 \begin{flushright} \large
     \emph{Responsable:} \\
     Geneviève HERY-ARNAUD
 \end{flushright}
\end{minipage}\\[2cm]
 
{\large \today}\\[8cm] % Date, change the \today to a set date if you want to be precise
 
% LOGO SECTION
\begin{minipage}[c]{0.3\textwidth}
   \includegraphics[width=0.7\textwidth]{img/logo_brest.jpg}\hfill
\end{minipage}
\begin{minipage}[c]{0.3\textwidth}
   \includegraphics[width=0.7\textwidth]{img/ubo.png}
\end{minipage}
%\begin{minipage}[c]{0.3\textwidth}
%        \includegraphics[width=0.7\textwidth]{img/logo_brest.jpg}     
%\end{minipage}
% 
 
 
 
\vfill % Fill the rest of the page with whitespace

\end{titlepage}


 %----------------PAGE ENGAGEMENT ET LICENSE ---------------------------

\newpage

\section*{Engagement de non-plagiat}

Je, soussigné Sacha SCHUTZ, interne en biologie moléculaire au CHRU de Brest, déclare être pleinement informé que le plagiat de
documents ou de parties de documents publiés sur toute forme de
support, y compris l'internet, constitue une violation des droits
d'auteur ainsi qu'une fraude caractérisée.

En conséquence, je m'engage à citer toutes les sources que j'ai
utilisées pour la rédaction de ce document.

Date : 17/05/2017

\vspace{0.5cm}

Signature : \\

ssh pub key fingerprint : a4:e3:da:87:78:2d:e1:6f:bb:56:5c:d1:72:f5:50:63
\vfill 

\section*{Licence}

\begin{wrapfigure}{R}{0.3\textwidth}
\includegraphics[scale=0.5]{img/gfdl.png}\hfill
\end{wrapfigure}

Copyright (c) 2015 SCHUTZ Sacha. Permission est autorisée de copier,
distribuer et/ou modifier ce document sous les termes de la Licence de
Documentation libre GNU, Version 1.2 ou toute version ultérieure publiée
par la Free Software Foundation ; with no Invariant Sections, no
Front-Cover Texts, and no Back-Cover Texts. Une copie de la licence est
incluse dans la section intitulée «``GNU Free Documentation License''.»

 %----------------PAGE REMERCIEMENT: A FAIRE.. ---------------------------
\thispagestyle{empty} 
\setcounter{page}{0}
\thispagestyle{empty} 

\newpage

\tableofcontents
\newpage


\section{Avant-propos}

Depuis Pasteur, les bactéries ont toujours été associées aux maladies. La mise en évidence des agents pathogènes comme la syphilis ou la peste n'a pas aider à tordre le cou de ce stéréotype. La médecine les a donc longtemps considéré comme l'ennemie à combattre plutôt qu'un allié. 
Aujourd'hui, personne ne peut nier que les traitements anti-infectieux ont permis l'amélioration générale de notre état de santé. 
Les avancées majeures dans les domaines de l'hygiène, la vaccination et les antibiotiques ont conduit à faire diminuer la prévalence des maladies infectieuses voire à les faire disparaître. Cependant, la destruction systématique et massive des micro-organismes qui font partie de nous depuis des milliers d'années pourrait bien être la cause de l'émergence de nouvelles maladies ( Figure \ref{hyigienisme} ).


\begin{figure}[ht]
\begin{center}
\includegraphics[scale=0.5]{img/allergie_infection.jpg}\hfill
\end{center}
\caption{Incidence des maladies infectieuses et auto-immunes en Europe au cours du temps \cite{Bach2002}. }
\label{hyigienisme}
\end{figure}


Les méthodes récentes d'exploration de ce monde microscopique comme le séquençage haut débit ont permis aux bactéries de retrouver leurs lettres de noblesse.
Elles sont présentes partout et jouent le premier rôle dans le fonctionnement des écosystèmes. Elles sont, par exemple, impliquées dans le cycle de l'azote et permettent à la biomasse d'absorber le diazote atmosphérique. Les bactéries sont ainsi la source primaire à laquelle les organismes puisent pour construire leurs protéines et leurs acides nucléiques.
Elles peuvent survivre dans les milieux les plus inhospitaliers. Les archées (anciennement archéobactéries) peuvent résister dans des environnements d'acidité et de température exceptionnellement élevées. On les retrouve dans les fonds océaniques où, privées de lumière, elles sont la seule source d'énergie pour la faune grâce à la chimiosynthèse à l'instar de la photosynthèse. 
Le corps humain est lui aussi un environnement riche en bactéries. La majorité des bactéries a longtemps été indétectable par les méthodes de culture classiques. Mais à présent, nous savons que les régions anatomiques autrefois considérées stériles contiennent beaucoup de bactéries. 
On les retrouve sur tous les territoires du corps exposés où elles se regroupent en communautés.
La peau est colonisée par \textit{Propionibacterium}, \textit{Corynebacterium} et \textit{Staphylococcus} \cite{Yong2017}. Le vagin est colonisé essentiellement par \textit{Lactobacillus} et la bouche principalement par \textit{Streptococcus}\cite{Yong2017}.
La muqueuse intestinale est colonisée par une "flore" bactérienne ou microbiote dominée par les anaérobies pouvant contribuer jusqu'à 0.2 kg au poids corporel \citep{Sender2016}.
En échange de son hospitalité, le microbiote contribue au bon fonctionnement de son hôte. Il aide à la digestion en dégradant par exemple les sucres du lait maternel chez le nouveau-né \cite{Bode2012,Yong2017}. Il participe à la synthèse des vitamines essentielles (K, B12, B9)\cite{LeBlanc2013,Yong2017}. Il éduque notre système immunitaire et fait barrière à l'implantation de tout nouvel agent pathogène.
Toute déséquilibre du microbiote intestinale ou \textit{dysbiose}, peut avoir des conséquences pour notre santé. La liste des affections associées est longue \cite{Yong2017}. On y retrouve la maladie de Crohn, la maladie coeliaque, le cancer de l’intestin, le syndrome du côlon irritable, l’obésité, le diabète de type 1, l’asthme, l’eczéma, la sclérose en plaque, la polyarthrite rhumatoïde, la maladie d’alzheimer et même l’autisme. \\
Par exemple, la destruction du microbiote intestinale par des antibiotiques donne l'opportunité à la souche toxinogène de \textit{Clostridium difficile} de s'installer et provoquer la colite pseudo-membraneuse. Un des traitements proposés après plusieurs échecs de traitement antibiotique est la transplantation fécale visant à régénérer le microbiote du patient. \\
Le microbiote nous amène donc à reconsidérer notre individualité. Nous ne sommes plus seulement un organisme multicellulaire composé d'un seul génome ; Mais plutôt un écosystème où cellules eucaryotes et micro-organismes vivent en symbiose. Cette relation n'étant pas figée dans le temps et pouvant varier entre le commensalisme, le parasitisme et le mutualisme. Des études récentes ont permis d'estimer pour un être humain le nombre  de cellules microbiennes à 39 milliards comparé aux 30 milliards de cellules humaines \cite{Sender2016}. En associant les gènes bactériens, le génome d’un être humain passe de 23 000  à 3,3 millions de gènes \cite{Qin2010} avec toute la complexité des interactions que cela engendre. Les scientifiques ont nommé \textit{holobionte} cet écosystème vivant. L'ensemble des génomes est appelé \textit{hologénome}. \\
Il faut toutefois rester prudent quant au rôle donné aux microbiotes et éviter de tomber dans l'excès. Nombreuses sont les publications scientifiques qui se contredisent ou qui confondent corrélation et causalité. Ces publications ont d'ailleurs conduit à la création du hashtag humoristique sur Twitter: \textit{\#GutMicrobiomeAndRandomSomething}. 
Les études sur le microbiote doivent être étayées par des études fonctionnelles afin de trouver des relations de cause à effet. Les corrélations doivent être réalisées sur des populations plus grandes avec un suivi dans le temps plus conséquent. Les nouvelles technologies de séquençage haut débit vont dans ce sens en collectant toujours plus de données.\\
Il est encore trop tôt pour dire si cette science va révolutionner la médecine ou s’ il s'agit d'un effet de mode. Au regard de l'évolution biologique et de l'importance du microbiote chez d'autre espèces, les paris sont ouverts. Ne l'oublions pas, ce sont d'anciennes bactéries, que nous appelons aujourd'hui des mitochondries (Figure \ref{mitochondrie}), qui permettent à l'ensemble de nos cellules de respirer. 

\begin{figure}[ht]
\begin{center}
\includegraphics[scale=0.5]{img/mitochondrie.jpg}\hfill
\end{center}
\caption{La mitochondrie est l'exemple de symbiose ultime entre eucaryote et procaryote}
\label{mitochondrie}
\end{figure}



\newpage

\section{Définition}
\subsection{Termes utilisés en écologie}

\begin{description}
\item[Le microbiote\cite{Eisen} ]est l’ensemble des micro-organismes (bactéries, levures, champignons, virus) vivant dans un environnement donné.
  
\item[Le microbiome\cite{Eisen}] est l'ensemble des génomes du microbiote. Le plus souvent le microbiome fait référence aux génomes bactériens seuls. On parle alors de virome et de mycobiome pour les virus et champignons respectivement.

\item[La biocénose] est le terme écologique dans un sens large désignant l'ensemble des organismes vivants dans un environnement appelé \textbf{biotope}. Biocénose et biotope forment ensemble un \textbf{écosystème}.

\item[Une symbiose] est une association durable entre deux organismes. Leurs relations peuvent être mutualiste, parasitaire ou commensale.


\item[L'OTU] (\textit{Operational Taxonomic Unit}) est un terme utilisé en phylogénie pour désigner un groupe d’individus semblable. Dans la majorité des cas, il s'agit de l'espèce dans la classification de Linné.
En microbiologie, un OTU est défini par un groupe de bactéries ayant une similarité dans leurs séquences d'ADNr 16S supérieur à 97\%.

\item[L'abondance] absolue est le nombre d'individus d'un même OTU dans un échantillon. L'abondance est indirectement évaluée en comptant le nombre de molécules d'ADNr 16S. 
L’abondance relative est l'expression en pourcentage de l'abondance absolue.

\item[La dominance] caractérise un OTU dont l'abondance est supérieur à 90\% dans un échantillon.


\item[La table des OTU] est à la base de toutes analyses en écologie. Elle correspond à un tableau à double entrée contenant l’abondance par échantillon et par OTU. Dans le tableau \ref{OTUTABLE}, l'abondance relative pour \textit{Staphylococcus aureus} est de 68\%.

\begin{figure}[h]
\begin{center}
\begin{tabular}{|l|c|c|c|c}
  \hline
   & échantillon 1 & échantillon 2 & échantillon 3  \\
  \hline
  OTU 1(\textit{S.aureus}) & 68\% & 12\% & 25\% \\
  OTU 2(\textit{E.coli})& 40\% & 24\% & 25\% \\
  OTU 3(\textit{P.aeruginosa}) & 28\% & 64\% & 50\% \\

  \hline
\end{tabular}
\end{center}
\caption{La table des OTUs : L'abondance relative de 3 espèces bactériennes pour 3 échantillons}
\label{OTUTABLE}
\end{figure}

\item[La diversité alpha] est une mesure de la biodiversité au sein \textbf{d’un} échantillon. Elle correspond à une colonne de la table des OTUs. La richesse, l'indice de Chao1, de Shannon et de Simpson sont des indices de diversité alpha.

\item[La diversité bêta] est une analyse descriptive de la biodiversité entre \textbf{plusieurs} échantillons. Elle correspond à l’ensemble de la table des OTUs. L’approche la plus courante est de réaliser une analyse multivariée par des méthodes d’ordination. Il s’agit de représenter un graphique à n dimensions, impossible à dessiner, en le projetant dans un espace à deux ou trois dimensions.

\item[La richesse] ( richness ) est le nombre d'OTU présent dans un échantillon. Les deux échantillons suivant ont la même richesse (2) mais pas les mêmes abondances.

échantillon 1  : 4 \textit{Streptoccus} , 4 \textit{Escherichia}  \\ 
échantillon 2 : 432 \textit{Streptoccus}, 12 \textit{Escherichia} 

\item[L'équitabilité] ( eveness )  indique si les OTUs d’un échantillon sont réparties uniformément.
L'équitabilité du premier échantillon est plus grande que le second

échantillon 1  : 50 Streptococus , 50 Escherichia  \\ 
échantillon 2 : 432 Streptoccus, 12 Escherichia 


\item[L'indice Chao1] est une estimation de la richesse réelle (\textit{in vivo}) par rapport à la richesse observée dans l'échantillon(\textit{in vitro}). Cet indice part du principe que si l’échantillon contient beaucoup de singletons (OTU détecté une seule fois), il est fort probable que la richesse réelle soit plus grande que la richesse de l’échantillon. La formule est la suivante:

\begin{mycapequ}[!h]
   \begin{equation}
     Chao1 = S_{observed} + \frac{a^2}{2b}
   \end{equation}
      \caption{avec \textbf{S} la richesse observée, \textbf{a} le nombre de singletons et \textbf{b} le nombre de doubletons}
\end{mycapequ}

\item[L'indice de Shannon] est un indicateur évaluant à la fois la richesse et l'équitabilité dans un échantillon. Il se calcule de la même façon que l’entropie de Shannon.

\begin{mycapequ}[!h]
   \begin{equation}
     Shannon = \sum_{i=1}^Sp_{i}ln(p_{i})
   \end{equation}
      \caption{avec \textbf{p} la fréquence d'un OTU parmi les \textbf{S} OTUs présents dans l'échantillon}
\end{mycapequ}

\item[L'indice de Simpson] est un indicateur évaluant la probabilité que deux individus sélectionnés aléatoirement dans un échantillon donné soient de la même espèce. L'indice est de sens contraire aux précédents. La formule est la suivante:

\begin{mycapequ}[!h]
   \begin{equation}
     Simpson = 1 - \sum_{i=1}^Sp_{i}
   \end{equation}
      \caption{avec \textbf{p} la fréquence d'un OTU parmi les \textbf{S} OTUs présents dans l'échantillon}
\end{mycapequ}

\newpage

\item[La courbe de raréfaction] est utilisée pour déterminer si la profondeur de séquençage est suffisante pour caractériser la diversité d’un échantillon. Pour comprendre imaginons un sac noir contenant des billes (reads) de différentes couleurs (espèces). Avec deux couleurs, une petite poignée ( un petit échantillon) de bille sera suffisante pour évaluer la diversité. Vous ne trouverez pas de nouvelle couleur avec plus de bille dans votre main. En revanche, avec d'avantage de couleurs dans le sac, il faudra une poignées plus importante pour évaluer la diversité. La courbe de raréfaction permet d'évaluer si cette poignée de bille est suffisamment représentative.
Pour générer cette courbe, des groupes de reads de taille croissante (1..n) sont tirés aléatoirement depuis l' échantillon. Le groupe est reporté sur l'axe X et le nombre d'OTU retrouvé dans ce groupe est reporté sur l’axe Y.
Une courbe s’aplatissant indique une bonne profondeur de séquençage ( Figure \ref{rarefaction_demo} ).
\end{description}

\begin{figure}[ht]
\begin{center}
\includegraphics[scale=0.5]{img/rarefaction_example.png}\hfill
\end{center}
\caption{Exemple de courbes de raréfactions. La courbe bleu témoigne d'un bon échantillonnage en s’aplatissant précocement. Plus de reads ne permettrait pas de découvrir de nouveau OTU contrairement à la courbe rouge. }
\label{rarefaction_demo}
\end{figure}




\subsection{Termes utilisés en bioinformatique}

\begin{description}
\item[Pipeline] 
Un pipeline est un ensemble d'étapes de calcul informatique. Chaque étape prend en entrée des fichiers pour en produire des nouveaux dans sa sortie. On peut comparer cela aux étapes d'une recette de cuisine. Sans parallélisation, un cuisinier (le processeur) doit attendre de faire fondre le beurre avant de battre les œufs en neige (exécution synchrone). En parallélisant, le cuisinier peut réaliser plusieurs étapes en même temps. Battre les œufs pendant que le beurre fond (exécution asynchrone). 
Maintenant si l'objectif est de produire 188 gâteaux (188 analyses) et que l'on dispose de 64 cuisiniers (64 processeurs), l'organisation des tâches devient complexe si l'on veut maximiser le rendement. Pour cela, on dispose d'outils comme \textit{Snakemake}\cite{Koster2012}, qui permettent de trouver la meilleure façon d'optimiser les tâches en construisant un Graphe orienté acyclique.

\item[NGS]( Next Generation Sequencing ) désigne l'ensemble des séquenceurs de nouvelle génération permettant de séquencer un nombre très important de fragments d'ADN. 

\item[La métagénomique] est une méthode d’étude du contenu en ADN présent dans un milieu grâce aux techniques de séquençage haut débit. Contrairement à la génomique qui s’intéresse au génome d’un individu, la métagénomique s’intéresse aux génomes d’une population d’individus.
Dans son sens strict, la métagénomique correspond à l’étude de l’ensemble des séquences d'ADN. L’analyse d’un seul gène, comme celui de l’ARN 16S est associé à tort au terme métagénomique, mais son usage reste courant. On lui préférera le terme de \textbf{metagénétique} ou \textbf{métataxonomique}.


\item[Un read] est un terme bio-informatique désignant une séquence d’ADN issue d’un séquençage haut débit. Selon les technologies, les reads varient entre 150 et 300 paires de bases.

\item[Une librarie] est l'ensemble des fragments d'ADN préparés pour être séquencé. 

\item[Score de qualité Phred] ou QScore exprime la confiance que l'on porte au séquençage. Il est lié de façon logarithme à la probabilité d'erreur d'identification d'un nucléotide.  Par exemple un score de 10 équivaut à une 1 erreur sur 100. Un score de 40 à 1 erreur sur 10000.


\begin{mycapequ}[!h]
   \begin{equation}
    Q = -10 \ \log_{10} P 
   \end{equation}
      \caption{Le score Q est associé de façon logarithmique à la probabilité d'erreur P de s'être trompé en séquençant un nucléotide}
\end{mycapequ}




\item[Une ordination] est une analyse multivariée visant à décrire des objets caractérisés par plusieurs variables. Par exemple un échantillon est caractérisé par différentes abondances de bactéries. Il existe plusieurs méthode d'ordination. La plus connus étant l'analyse en composante principale (PCA) basé sur une matrice de covariances. En écologie l'analyse en Coordonnée principale ( PCoA ) est préférée. Elle se base sur une matrice de distances. Elle consiste à projeter des points d'un espace à n dimensions vers un espace visible en 2 dimensions.
La figure \ref{pcoatuto} illustre ce principe avec un exemple à deux dimensions seulement. 

\begin{figure}[!h]
\begin{center}
\includegraphics[scale=0.6]{img/pcoatuto.png}\hfill
\end{center}
\caption{Représentation d'un échantillon ne contenant que deux espèces A et B dans un espace à deux dimensions. Chaques échantillons est assimilable à un point de coordonnées ( abondance A; abondance B ). Il est alors possible de calculer des distances entre les échantillons. La plus connus étant la distance euclidienne. Mais d'autres existent comme la distance de Bray-Curtis. Sur cette figure, l’échantillon 1 et 2 sont ont des microbiotes proches car la distance d1 qui les sépare est petite. Dans la réalité, il y a plus d'espèces donc plus de dimensions. Une PCoA consiste à réduire cette espace multidimensionnel en un espace 2D respectant les distances }
\label{pcoatuto}
\end{figure}
\newpage

\item[La distance de Bray-Curtis] est un indice de dissimilarité entre deux échantillons qui s'assimile à une distance entre 2 points dans un espace de taille n ( Figure \ref{pcoatuto} ). Plus la distance est grande et plus les deux échantillons sont dissemblable. La formule est la suivante : 

\begin{mycapequ}[!h]
   \begin{equation}
    BC_{jk} = 1 - \frac{2\sum_{i=1}^{p}min(N_{ij},N_{ik})}{\sum_{i=1}^{p}(N_{ij} + N_{ik})} 
   \end{equation}
      \caption{Où Nij est l'abondance d'une  espèce i dans l'échantillon j et Nik l'abondance de la même espèce i dans l'échantillon k. Le terme min(.,.) correspond au minimum obtenu pour deux comptes sur les mêmes échantillons. Les sommes situés au numérateurs et dénominateur sont réalisées sur l'ensemble des espèces présentes dans les échantillons.}
\end{mycapequ}


\item[Un fichier fastq] contient les séquences des reads et leurs qualités par nucléotides exprimée en Phred Score.

\begin{figure}[!h]
\begin{center}
\includegraphics[scale=0.6]{img/fastq.jpg}\hfill
\end{center}
\caption{Exemple d'un read contenu dans un fichier fastq. Les scores de qualités sont encodés avec des symboles ASCII. Par exemple le quatorzième T  est associé au symbole « : » correspondant au score phred de 25, soit une probabilité de 0.003\% d'erreur.  }
\label{fastq}
\end{figure}

\end{description}

\newpage

\setcounter{page}{1}

\section{Introduction}
\subsection{La mucoviscidose}
\subsubsection{Une maladie génétique}
La mucoviscidose est une maladie génétique autosomique récessive grave qui frappe en France 1 naissance sur 5400\cite{Registredelamuco.org}. La Bretagne est la région la plus touchée avec une prévalence de 1/3000\cite{Registredelamuco.org}.
La loi de Hardy Weinberg estime qu’en Bretagne 1 patient sur 25 est porteur de la mutation à l’état hétérozygote. Cette haute prévalence s’explique probablement par un effet fondateur associé à un avantage sélectif pour les individus porteurs de l’allèle muté. \footnote{Plusieurs hypothèses ont été proposées, notamment lors des grandes épidémies de choléra en diminuant les pertes hydriques.]} 
Le gène \textit{CFTR} impacté se situe sur le chromosome 7 en position q31.2. Il est constitué de 27 exons pour 250 188\cite{OLeary2016} paires de bases. Il code pour un canal chlore AMP dépendant permettant les échanges des ions chlorures au niveau des membranes cellulaires. Il est également impliqué dans le transport du thiocynate (SCN-) et des bicarbonates (HCO3-)\cite{Quinton2001}. 
On dénombre à ce jour 2017 mutations \cite{cftrdb} mises en cause dans la mucovicidose. La perte d’une phénylalanine en position 508 par délétion du est triplet c.1521-1523delCTT (anciennement $\Delta$F508) cause à elle seule 80\% des mucoviscidoses\cite{cftrdb}.
Ces mutations sont responsables d’une protéine défectueuse ou d’une absence de canaux sur les membranes cellulaires. \\
Cliniquement, la mutation entraîne une insuffisance pancréatique exocrine et une infertilité par disparition des canaux déférents. Des signes digestifs, hépatiques et articulaires sont également retrouvés.
L'atteinte de la fonction respiratoire est la plus bruyante. En effet au niveau de l’épithélium broncho-pulmonaire, l’absence d’un CFTR fonctionnel est responsable d’une déshydratation du mucus le rendant plus visqueux et empêche les cils bronchiques de jouer leurs rôles.\\
La forte prévalence de la maladie nécessite de réaliser un dépistage précoce chez tous les nouveaux nés (test de Gutri) afin d’adapter au plus tôt la prise en charge. Seul le test à la sueur permet de poser le diagnostic. Le dépistage prénatal basé sur l’ADN circulant est actuellement à l’étude.\\   
Le traitement repose avant tout sur une prise en charge respiratoire (kinésithérapie, dorase \footnote{Est une DNase qui fluidifie les sécrétions trop riche en ADN provenant de l'afflux des neutrophiles in situ et de leur apoptose prématurée}, antibiothérapie). Les recherches en thérapies génétiques sont encourageantes [@].
Les traitements correcteurs/potentiateurs de la protéine CFTR, comme l'\textit{ivacaftor} sont les seuls traitements curateurs. Mais concerne uniquement certaine mutation rare comme la G551D. La greffe pulmonaire est le dernier recours.

\subsubsection{Une maladie infectieuse et inflammatoire}

L’atteinte pulmonaire dans la mucoviscidose est caractérisée par des réactions inflammatoires qui dégradent progressivement la fonction respiratoire. Plusieurs pathogènes sont impliqués. Chez les jeunes enfants\cite{Davies}, \textit{Haemophilus influenza} et \textit{Staphylococcus aureus} sont le plus souvent responsable. \textit{Burkolderia cepacia complex} et \textit{Stenotrophomonas maltophilia} sont retrouvés parmi les sujets plus âgé \cite{Davies}.
Mais c’est \textit{Pseudomonas aeruginosa} qui caractérise l’atteinte pulmonaire dans la mucoviscidose en marquant un tournant décisif dans l’évolution de la maladie. Ce bacille aérobie stricte est une bactérie de l'environnement rarement retrouvé parmi les patients sains[@]. En revanche, dans la mucoviscidose, il est mis en évidence chez 60\%\cite{LeBourgeois} des patients jeunes, et plus de 90\% des patients adultes\cite{LeBourgeois}.\\
La primocolonisation à \textit{Pseudomnas aeruginosa} est difficilement détectable, mais semble avoir lieu tôt dans l’enfance[@]. Il y a ensuite une phase de latence, variable entre les individus, marquée par des épisodes d’exacerbations et de rémissions. À ce moment, l’éradication par des antibiotiques reste possible.
Puis survient le passage à la chronicité. \textit{Pseudomnas aeruginosa} s'adapte à son milieu et s’installe à long terme. Il perd certains caractères de virulence, mais devient résistant aux antibiotiques\cite{LeBourgeois}. Son phénotype change. Il se transforme pour devenir mucoïde en sécrétant un film d’alginate qui le protège du système immunitaire. Les mécanismes sous-jacents à cette adaptation sont ingénieux. La forte densité en bactérie est responsable d’activation de certains gènes par un processus appelé \textit{quorum sensing}\cite{Ruimy2004}. Un processus dans lequel chaque bactérie communique avec ses voisines par des signaux chimiques.
Le génome de \textit{Pseudomnas aeruginosa} devient aussi hyperpermutable\cite{Davies} afin de présenter une plus grande diversité génétique au regard de la sélection naturelle\footnote{En biologie évolutive, il s'agit d'évolvabilité}. \\
À ce stade, le traitement antibiotique n’est plus curatif et l'évolution tend inexorablement vers un déclin de la fonction respiratoire. \\
L'approche clinique est donc préventive afin de retarder le passage à la chronicité. Elle vise à éliminer \textit{Pseudomnas aeruginosa} dès qu'il est détecté en culture. Une surveillance rapprochée des patients avec un prélèvement mensuel ou bimensuel est préconisée selon l'HAS[@]. La culture étant peu sensible, d’autres méthodes d'identification peuvent être employées. La détection des anticorps anti-pyocyaniques par ELISA a montré peu de sensibilité \cite{Plesiat}.
La PCR ciblée associant les protéines bactériennes \textit{OPRL1} , \textit{GYRB1} et \textit{ECFX1}  s’est montrée plus sensible et plus spécifique que la culture\cite{Vongthilath2017,LeGall} \\
En pratique, la colonisation chronique est définie lorsque 3 expectorations sont rendues positives en culture, successivement au cours d’un suivi mensuel ou bimensuel \cite{LeBourgeois}.
Une autre classification, celle de Lee\cite{Lee2003} a montré une forte liaison clinico-biologique. Elle est composée de 4 groupes :  
\begin{description}
\item[groupe chronique] > 50\% des cultures sont positives sur 12 mois
\item[groupe intermédiaire] $\leq$ 50\% des cultures sont positives sur  12 mois
\item[groupe Free] Toute culture négative sur 12 mois, avec des antécédents
\item[groupe Nevers] Toute culture négative sur 12 mois sans antécédents 
\end{description}

On ne sait pas aujourd’hui pourquoi \textit{Pseudomnas aeruginosa} s’installe préférentiellement chez les patients atteints de mucoviscidose. Plusieurs hypothèses ont été proposées : 

%----------------- REFORMULER -----------------
\begin{itemize}
\item La dysfonction ciliaire empêche les pseudo d’être viré par le haut
\item L’hypersalinité du film muqueux désactive les peptides antimicrobiens
\item Le CFTR est un récepteur de pyo qui les internalise et les viré
\item L’inflammation de  l’epithelimum augmente les métabolites qui permettent de se développer.
\item Alanine et l’acta sont une source de carbone pour le pyo.
\item Le microbiote influence la colonisation
\item Le pyo stimule le Systeme immunitaire pour virer tous les autres concurrents 
\end{itemize}




\begin{figure}[ht]
\begin{center}
\includegraphics[scale=0.8]{img/chronic.png}\hfill
\end{center}
\caption{Infection pulmonaire à pyo dans la mucoviscidose. ref []}
\label{bach}
\end{figure}


\subsection{Le microbiote pulmonaire}

Bien qu'il soit en contacte avec le milieu extérieur, l’arbre respiratoire (comprenant la trachée, les bronches et les alvéoles) a longtemps été considéré stérile avec les méthodes de culture classique. Il a fallu attendre l’avènement du séquençage haut débit pour mettre en évidence le microbiote pulmonaire \cite{HoMan2017,Beck,Dicksonb}.\\
L'arbre respiratoire chez le foetus\footnote{Un article récent critique l'idée d'un foetus stérile.} est stérile. Il se colonise comme l'intestin, à la naissance en traversant la filière génitale puis se différencie avec le temps. Il est beaucoup moins riche que le microbiote digestif mais plus dynamique en raison d'un flux aérien bidirectionnel.  Les bactéries du microbiotes proviennent de l’air ambiant, des voies supérieures,  mais aussi du tube digestif par des micro-aspirations \cite{Dickson}. Une étude à d'ailleurs montrer une concordance entre le microbiote digestif et respiratoire\cite{Dickson}.\\
Le microbiote pulmonaire est dominé par le phylum des \textit{Firmicutes} (\textit{Streptococcus}) et des \textit{Bacteroidetes} (\textit{Prevotella}). Les genres retrouvés majoritairement sont \textit{Streptococcus}, \textit{Prevotella}, \textit{Fusobacteria}, \textit{Veillonella}, \textit{Haemophilus}, \textit{Neisseria} et \textit{Porphyromonas}.
L’arbre respiratoire étant en continuité direct avec les voies aériennes supérieur, certains genres bactériens sont communs, comme \textit{Streptococcus}, \textit{Staphylococcus}, \textit{Haemophilus} et \textit{Moraxella}. Tandis que d’autres genres comme \textit{Corynebacterium} et \textit{Dolosigranulum} ne sont retrouvés qu’au niveau du nez et de l'oropharynx. \\
Le microbiote pulmonaire varie dans l’espace et le temps. \\
Dans l’espace, du fait de sa structure, certaines régions de l'arbre bronchique peuvent présenter des microbiotes différents. Un prélèvement au niveau d'un foyer infectieux se distinguera nécessairement d'un foyer sain. Il existe également des différences physico-chimiques selon la région pouvant sélectionner des espèces. Les cavernes tuberculeuses par exemple se trouvent essentiellement dans le lobe supérieur en raison d'une concentration en oxygène plus élevée qui favorise ce bacille aérobie stricte. \\
Le microbiome varie dans le temps. Avec l'age la diversité diminue. 
Declin de la diversité alpha avec l'age .... probablement a cause des exacerbations multiple et du traitement
Le microbiote est variable entre les individus ... \\
Le microbiote est corrélé à la pathologie respiratoire. Plusieurs études suggèrent une différence entre patients sains et patients asthmatiques, BPCO ou atteint de mucoviscidose ...... [@].
Plus riche avec la maladie
Il est par exemple demontré que chez les jeunes enfant, une diminution d'infection réspiratoire est associé à une augmentation d'asthme. Une étude montre une augmentation des proteobacteria en echange d'une baisse des bacteroidetes ( haemo, maraexalla et neisseria).
Chez les muco, augmentation de la richesse et diminution de la diversité 
Chez les muco, augmentation des germes opportuniste par rapport au commensaux ( data mining paper)

Le microbiote pulmonaire joue probablement le rôle de barrière vis à vis d'autre agent pathogène. \cite{HoMan2017}

\subsection{Exploration du microbiote pulmonaire}
%(http://bacterioweb.univ-fcomte.fr/bibliotheque/remic/08-Bronc.pdf) ==> A LIRE COOURS BACTERIO

\subsubsection{Méthode de prélèvement}
Le microbiote respiratoire est exploré en séquençant l'ADN des micro-organismes présent dans un échantillon. 
Toutes les méthodes de recueils sont possibles, mais les prélèvements protégés (combicath, Liquide Broncho Alvéolaire (LBA)) sont recommandés afin
d’éviter une contamination par les voies aériennes supérieures. En pratique, les patients atteints de mucoviscidose sont suivis par le recueil des prélèvements moins invasifs comme les prélèvements pharyngées ou les expectoration spontannées ou induites (ECBC). Dans ces derniières, la qualité du prélèvement peut être évaluée en comptant le nombre de cellules épithélium ( normalement< 25 par champs à l'objectif x10 ) et de polynucléaire ( normalement > 10 par champ ç l'objectif x10) selon le score de Bartlett (REMIC 2015). Des prélèvements sont également réalisables \textit{in situ}  sur les explants lors des greffes pulmonaires.

\subsubsection{Séquençage haut débit}
% Baisse du coup du sequencage ... %
Le séquençage de nouvelle génération permet de séquencer de grande quantité d' ADN dans un échantillon et ainsi déterminer sa composition en bactéries ou autres micro-organismes. À titre d'exemple, un séquenceur Sanger classique permet de lire des fragments d'ADN d'environ 800 pb parallélisable jusqu'à 96 fois en augmentant le nombre de capillaires. 
À l'inverse, un séquenceur de nouvelle génération lit des fragments plus courts de l'ordre de 150pb. Mais la parallélisation peut aller jusqu'à 20 milliards de fois en un seul run sur un illumina Novaseq. \\
2 stratégies de séquençages sont utilisées en écologie microbienne:  \\
\textbf{La stratégie shotgun} consiste à séquencer l'ensemble des  ADN présents dans l'échantillon sans discernement, que ce soit humain ou bactérien. Les séquences sont filtrées puis les génomes bactériens sont reconstruits par des méthodes bio-informatiques complexes. \\
\textbf{La stratégie Amplicon} est moins coûteuse sur le plan de l'analyse. Elle consiste à séquencer un gène spécifiquement bactérien et suffisamment variable pour discriminer une espèce. Il s'agit du gène de l'ARN 16S.\\
L'ARN 16S est un ARN non codant participant à la structure de la petite sous unité des ribosomes bactériens. Il est composé de 1500 nucléotides et forme plusieurs boucles dans sa structure secondaire (Figure \ref{ARN16S}). 
L'alignement des séquences d'ARN 16S entre plusieurs espèces met en évidence des régions constantes et 9 régions variables (Figure \ref{ARN16SVariation}). Les régions constantes permettent de concevoir des amorces s'hybridant sur tous les ADNs bactériens. Les régions variables apportent la spécificité taxonomique permettant d'identifier l'espèce bactérienne.
% Les études de [@] ont montrée que certaine région variable apportent plus de spécificité.

\begin{figure}[ht]
\begin{center}
\includegraphics[scale=0.8]{img/ARN16S_variation.png}\hfill
\end{center}
\caption{Régions constantes et variables de l'ADNr 16S}
\label{ARN16SVariation}
\end{figure}

\begin{figure}[ht]
\begin{center}
\includegraphics[scale=0.8]{img/ARN_16S.png}\hfill
\end{center}
\caption{Structure secondaire de l'ARN 16S}
\label{ARN16S}
\end{figure}


En séquencant l'ensemble des génomes bactériens, la stratégie \textit{shotgun} est plus informative car elle permet de prédire la fonction d'un microbiote. En effet, les transferts génétiques horizontaux amènent à dissocier l'espèce de sa fonction. Deux bactéries d'une même espèce peuvent avoir des fonctions différentes. L'inférence fonctionnelle à partir de la deuxième stratégie est possible mais déconseillée.  
La stratégie 16S reste toutefois une méthode simple pour décrire les populations bactériennes présentes. C'est cette stratégie qui a été utilisée dans notre étude. 



\subsection{Objectif de l'étude Mucobiome}
L'objectif général de l'étude MUCOBIOME est de savoir si dans la mucoviscidose, le microbiote respiratoire influence la colonisation à \textit{Pseudomnas aeruginosa}. 
Pour cela, nous avons suivi pendant 3 ans,  une cohorte de 47 patients atteints de mucoviscidoses exempts de Pseudomonas aeruginosa depuis au moins 1 an.
L'exploration de leur microbiote par la stratégie 16S a été réalisée à partir d'expectorations bronchiques obtenues dans le cadre de leurs suivis longitudinal. 
À partir des données générées par le séquençage, nous avons effectué l'étude descriptive et analytique de leurs microbiotes en créant un pipeline bio-informatique dédié. L'objectif principal de ce travail a été la mise au point et l'évaluation de ce pipeline à façon pour l'étude MUCOBIOME.

\section{Matériel et Méthodes}
\subsection{Recueil des données}

Quatre-vingt dix-sept patients atteints de mucoviscidose ont été suivis sur 3 ans (2008-2011) dans une étude prospective multicentrique (Nantes, Brest, Roscoff) appelée \textit{Mucobiome}.
Le comité de protection des personnes ( CPP VI-Ouest ) et le comité d’éthique du CHRU de Brest ont approuvé le protocole. Tous les patients ( ou les parents pour les mineurs) ont signé un consentement éclairé. Le protocole a fait l’objet d’une déclaration de biocollection à l’ARS et au MESR (n DC-2008-214).\\
Les ECBC des patients ont été recueillies lors des séances de kinésithérapie respiratoire tous les 3 mois, suivant le calendrier des recommandations officielles. En pratique, sur l’ensemble de la cohorte suivie, l’intervalle médian entre 2 consultations a été de 3,4 mois.
Les patients devaient avoir un génotypage CFTR et un test à la sueur positif. Les transplantés ont été exclus de l’étude.
Pour être inclus, le patient devait être exempt de P. aeruginosa depuis au moins un an d’après les résultats de la culture bactérienne des expectorations. Le patient devait être en capacité d’expectorer spontanément (exclusion des prélèvements pharyngés).
La première culture bactérienne positive à PA était le « endpoint » de l’étude. A cette étape le patient était sorti de l’étude pour être ré-inclus un an plus tard en cas de maintien de la négativité des cultures bactériennes obtenue par l’antibiothérapie d’éradication anti-P. aeruginosa. D’autres données clinico-biologiques ont également été recueillies. Au total, 756 expectorations ont été recueillies à l’issue de l’étude MUCOBIOME. Pour ce travail portant sur l’étude du microbiote broncho-pulmonaire, 188 ont été sélectionnées.

citer qq exemples de données (que tu exploiteras après) et préciser le score
cytologique (à mettre dans les résultats : impact oui ou non sur les données du microbiote)

\begin{figure}[ht]
\begin{center}
\includegraphics[scale=0.8]{img/summary.png}\hfill
\end{center}
\caption{Données associés par échantillons ( à refaire par patients)}
\label{summary}
\end{figure}



\subsection{Extraction de l’ADN}

La procédure d'extraction des ADN des échantillons est détaillée dans la publication de Le Gal... Brièvement les échantillons ont été liquéfiés avec du Dithiotréitol( DTT). Les protéines ont été dégradées avec de la protéinase K.
Les parois bactériennes ont été fragmentées par sonication. (DTT par sonication (Elamsonic S10, Singen, Germany). Après 10 min de centrifugation, L’ADN a été extrait à partir du culot à l'aide du kit QUIAamp DNA Minikit ( Quagen).
Les extraits d’ADN ont été envoyés pour séquençage par un prestataire GATC.

\subsection{Séquençage}
La librairie a été générée en amplifiant la région V3-V5 à l’aide du couple d’amorces  \textit{forward(CCTACGGGAGGCAGCAG)} et \textit{reverse(CCGTCAATTCMTTTRAGT)} et du kit MiSeq Reagent Kits v3. \\
Le séquençage a été réalisé sur Illumina MiSeq pour produire un couple de séquences chevauchantes de 300 pb (Figure \ref{illumina}). Après fusion du couple, le séquençage permet de lire  535 pb correspondant à la région V3-V5.\\
Environ 25 millions de reads sont produits par run MiSeq. En multiplexant à l’aide de 94 index, les 188 échantillons ont été séquencés sur 2 runs pour produire 188 x 2 fichiers fastq.

\begin{figure}[ht]
\begin{center}
\includegraphics[scale=0.6]{img/illumina.png}\hfill
\end{center}
\caption{le couple de séquence de 300pb permet de recouvrir l'ensemble de la région V3-V5 de l'ARN 16S}
\label{illumina}
\end{figure}


\subsection{Analyse bio-informatique}
L’analyse des 188 pairs de fichiers fastq a été réalisée grâce à un pipeline bio-informatique, appelé \textit{mucobiome}, conçu et testé dans le cadre de cette étude. Par rapport aux autres logiciels comme \textit{QIIME}\cite{Caporaso2010} ou \textit{MOTHUR}\cite{Schloss2009}, le pipeline mucobiome est spécialisé dans l’analyse des données 16S. Il est également plus rapide en raison d’un très haut niveau de parallélisation permis grâce à  \textit{Snakemake}\cite{Koster2012}. Cet outil modélise l'ensemble du pipeline sous forme d'un graphe direct acyclique (DAG) et le résout afin d'optimiser la parallélisation. \\
Le pipeline mucobiome prend en entrée, les 188 couples de fichiers fastq provenant du séquençage et produit un fichier BIOM contenant la table des OTUs. Les figures \ref{pipeline_trio} et \ref{dag} illustrent toutes les étapes du pipeline qui sont décrites ci-dessous.



\begin{figure}[!ht]
\begin{center}
\includegraphics[scale=0.5]{img/pipeline_trio.png}\hfill
\end{center}
\caption{Graphe du pipeline simplifié sur 3 échantillons 1015, 1001 et 1003.\\ \textbf{merging}: Les reads pairs de 300 pb sont fusionnés  pour produire un fichier fastq contenant des reads de 535pb. \textbf{cleanning}: les reads de mauvaise qualité sont supprimés. \textbf{reversing}: les reads sont transformés en leurs séquences complémentaires pour pouvoir être alignés. \textbf{Trimming}: seule la séquence entre les primers V3-V5 est conservée. \textbf{dereplicating}: les séquences dupliquées sont retirées. \textbf{merging}: L'ensemble des séquences est regroupé dans un seul fichier. \textbf{Taxonomy assignement}: Les séquences sont alignées sur la base de données greengene. \textbf{create\_biom}: la table des OTU est créée }
\label{pipeline_trio}
\end{figure}


\begin{figure}[ht]
\begin{center}
\includegraphics[scale=0.4]{img/dag.jpg}\hfill
\end{center}
\caption{le graphe du pipeline sur l'ensemble des échantillons}
\label{dag}
\end{figure}

% +++ RELECTURE 
\subsection{Étape du pipeline}
\subsubsection{Merging: Fusions des reads}\begin{center}\emph{ 2 fastq en entrée 1 fastq en sortie. } \end{center}

La première étape du pipeline consiste à fusionner le couple de fichier fastq afin de produire un seul fichier contenant des plus longues séquences de 535pb correspondant à la région V3-V5 de l’ARN 16S.
Le programme \textbf{Flash}[@] a été utilisé avec les paramètres par défaut. À partir de deux fichiers fastq, ce dernier recherche le meilleur alignement entre deux reads et produit 1 fichier fastq contenant les reads fusionnés.
Une analyse qualitative des reads a été réalisée avec FastQt\citep{Beck} avant et après fusion.

\subsubsection{Cleaning: Filtrage des qualités}\begin{center}\emph{1 fastq en entrée 1 fastq en sortie. } \end{center}

Les données de séquençage haut débit peuvent contenir beaucoup d'erreurs. Il est important de supprimer les reads de mauvaise qualité pour gagner en spécificité. ( expliquer la qualité).
Le filtrage des reads de mauvaise qualité est réalisé avec le programme \textbf{sickle}. Son algorithme repose sur l'utilisation d'une fenêtre glissante de taille définie ( par défaut : 20 Pb). Cette fenêtre glisse le long de la séquence et pour chaque position calcule la moyenne des scores de qualité dans cette fenêtre. Si successivement le score moyen passe sous un certain seuil, le read est supprimé. Les paramètres utilisés sont ceux par défaut. Un score de 20 avec une fenêtre glissante de 20pb. 
Une analyse qualitative des reads a été réalisée avec FastQt\citep{Beck} après le filtrage. 


\subsubsection{Reversing: Séquence complémentaire}\begin{center}\emph{1 fastq en entrée 1 fasta en sortie. } \end{center}

Les reads produits par le séquenceur ne sont pas orientés dans le même sens que la base de données \textit{greengene}. Pour permettre l'alignement, les séquences ont été remplacées par leurs séquences complémentaires grâce au programme \textbf{seqtk}. 
Par la même occasion les scores de qualités devenus inutiles sont supprimés. Les séquences sont sauvegardées dans un fichier fasta. 

\subsubsection{Trimming: Suppression des primers} \begin{center}\emph{1 fasta en entrée 1 fasta en sortie. } \end{center}

Pour permettre un alignement parfait entre les reads et la base de données, les primers sont retirés et seule la séquence V3-V5 est conservée. Cette étape est réalisée aussi bien pour les données du séquençage que  la base de données greengene.  
Le programme \textbf{cutadapts} est utilisé avec une tolérance de 0.1 par défaut. 

\subsubsection{dereplicating: Suppression des doublons}\begin{center}\emph{1 fasta en entrée 1 fasta en sortie. } \end{center}

Cette étape consiste à supprimer toutes les séquences doublons. En procédant ainsi, on s'assure de ne pas répéter l'assignement taxonomique plusieurs fois sur des reads identiques. Le nombre de reads dupliqués est conservé pour être pris en considération lors du calcul des abondances. Il s'agit d'une étape d'optimisation permettant d'économiser en temps de calcul. La déréplication a été réalisé avec \textbf{vsearch} et sa fonction \textit{--derep\_fulllength }

\begin{figure}[ht]
\begin{center}
\includegraphics[scale=0.4]{img/dereplication.png}\hfill
\end{center}
\caption{Exemple de déréplication d'un fichier fasta}
\label{dereplication}
\end{figure}

\subsubsection{Assignement taxonomique} \begin{center}\emph{2 fasta en entrée,  1 fichier biom  en sortie. }\end{center} 

L’assignement taxonomique consiste à labelliser chaque read à son taxon. Nous avons utilisé la stratégie \textit{close référence} dont le rôle est de comparer chaque read à une base de données avec un seuil de 98\% de similarité. Cet algorithme est de complexité N. C'est-à-dire que le temps de calcul est directement proportionnel au nombre de reads testé. La base de données \textit{Greengene} version mai 2013  a été utilisé. Il s'agit d'un fichier fasta contenant 1 262 986 séquences et 203 452 OTUs. \\
L'autre stratégie d'assignement \textit{de novo} n'a pas été utilisée. Cette dernière, de complexité $N^{2}$, consiste à comparer les reads entre eux pour former des groupes. Elle s'emploie de préférence pour détecter les bactéries absentes des bases de données. \\
L'assignement taxonomique a été réalisé avec vsearch et sa fonction \textit{--usearch\_global}. 

\subsubsection{Analyse de la table des OTUs}
L’analyse de la table des OTUs a été réalisée avec R et le package phyloseq. Les OTUsont été regroupé par genre bactérien. L’étude descriptive à été réalisé en calculant desindicateurs de diversité alpha et en produisant différents graphiques. Le core microbiotaa été calculé. Il est défini comme l’ensemble des taxons retrouvé dans plus de 50% deséchantillons avec une abondance supérieur à 0,1%. Les categories Free et Never ont étécomparé d’après leurs diversité alpha et par une analyse multivarié de leurs microbiom


\section{Résultat}
\subsection{Séquençage et pipeline}
\subsubsection{Données de séquençage}
Après démultiplexage, 188 x 2 fichiers Fastq ont été générés soit 2 fichiers par échantillons.
La longueur des reads dans l'ensemble est de 301 paires de bases.
Au total, 115 002 297 reads ont été produits sur 2 runs MiSeq (Figure \ref{readcount}). Avec en moyenne 616 900 reads par échantillon. Un minimum de 61 422 reads pour l’échantillon 2154 et un maximum de 1 071 188 pour l’échantillon 3165. 


\begin{figure}[ht]
\begin{center}
\includegraphics[scale=0.25]{img/pipeline.png}\hfill
\end{center}
\caption{Nombre de reads par échantillons}
\label{readcount}
\end{figure}


\subsubsection{Qualité des reads}
La figure \ref{fastqt} montre la qualité typique d'un fichier fastq produit par le séquenceur. Une baisse de qualité importante est observée à hauteur du 250ème nucléotides. Tous les fichiers fastq présentaient le même profil. 
A cause de cette baisse de qualité, x\% des reads pairs n'ont pas réussi à fusionné. 
L'étape de filtrage a permis de ramener la qualité médiane au dessus de 20 ( Figure \ref{fastqt_after} ).
Au total seulement 49.24\% des reads sont conservé pour l'analyse ( Figure \ref{readcount} ) avec des bornes allant de 37,30\% à 61,13\%.



\begin{figure}[ht]
\begin{center}
\includegraphics[scale=0.45]{img/1003_forward.png}\hfill
\end{center}
\caption{Qualité par nucléotide des reads forward de l'échantillon 1003. \textbf{Axe X}: la position sur le read. \textbf{Axe Y}: La distribution des qualités}
\label{fastqt}
\end{figure}


\begin{figure}[h]
\begin{center}
\includegraphics[scale=0.45]{img/duo_merging.png}\hfill
\end{center}
\caption{Qualité par nucléotide des reads fusionnés pour l'échantillon 1003}
\label{fastqt_after}
\end{figure}


\subsubsection{Assignement taxonomique}
99.88\% (4615960) des reads analysables ont reçu une assignation taxonomique pour 10517 OTUs \footnote{Pour chaque espèce, il y a plusieurs OTUs définis dans greengene} correspondant à 54 genres bactériens ( Figure \ref{readgenus} ). 
Le temps de calcul a été de 1 h 29 et a nécessité 40 cœurs et 20 gigaoctets de mémoire contre 21h de calcul sans l’optimisation par déréplication.


\begin{figure}[h]
\begin{center}
\includegraphics[scale=0.5]{img/read_count_genus_all.png}\hfill
\end{center}
\caption{Nombre de reads par genre bactérien}
\label{readgenus}
\end{figure}


\subsubsection{Profondeur d’échantillonnage}
La figure \ref{rarefaction} montre les courbes de raréfaction pour les 188 échantillons.
Dans l'ensemble elles s’aplatissent précocement, témoignant d’un bon niveau d'échantillonnage. Les quelques échantillons n'ayant pas attend l'asymptote horizontal ont tout de même été conservés . 

\begin{figure}[!h]
\begin{center}
\includegraphics[scale=0.5]{img/rarefaction.png}\hfill
\end{center}
\caption{Courbe de rarefaction}
\label{rarefaction}
\end{figure}


\begin{figure}[!ht]
\begin{center}
\includegraphics[scale=0.5]{img/phylum.png}\hfill
\end{center}
\caption{Phylum}
\label{phylum}
\end{figure}

\subsection{Résultats descriptifs}
\subsubsection{Composition du microbiote}
Cinquante-quatre genres ( Figure \ref{readgenus} ) et sept phylums ( Figure \ref{phylum} ) bactériens sont retrouvés dans l'ensemble des échantillons analysés. 
Les deux phyla majoritaires sont \textit{Firmicute} ( 42.59\%) et \textit{Proteobacteria} ( 31.48\%). Parmi les \textit{Firmicutes} majoritaires, on retrouve \textit{Streptococcus} et \textit{Staphylococcus}. Chez \textit{Proteobacteria}, \textit{Neisseria} et \textit{Haemophilus} sont les genres les plus abondantes.
Le tableau de la figure \ref{alltable} résume l'ensemble des résultats en y associant la prévalence des genres bactériens parmi les 188 échantillons.
Par exemple \textit{Streptococcus}, \textit{Neisseria}, \textit{Prevotella}, \textit{Granulicatella}, \textit{Gemella}, \textit{Veillonella} et \textit{Fusobacterium} sont très prévalents, car présents dans plus de 185 échantillons.
D’autres genre sont dominants. Il s’agit de \textit{Streptococcus}, \textit{Neisseria}, \textit{Haemophilus} et \textit{Staphyloccoccus}. \textit{Sténotrophomonas} et \textit{Achromobacter} sont retrouvées dans 64 et 8 échantillons respectivement. Pseudomonas est retrouvé dans 53 échantillons, dont un, au moins avec une abondance de 25.94\%. Burkholderia est retrouvé seulement dans deux échantillons à moins de 1\%.\\
Le core microbiota est défini comme l'ensemble des taxons retrouvé dans plus de 50\% des échantillons avec une abondance supérieur à 0,1\%. Il est constitué de 15 genres bactériens ( Figure \ref{core} ).\\
La distribution du core microbiota est illustré dans la figure \ref{violon}.
\textit{Streptococcus} respecte grossièrement une distribution normale variant de la quasi-absence à la dominance avec une moyenne de 30\% par échantillon. Neisseria est le deuxième genre le plus abondant avec une moyenne de 18\%.
Les abondances de \textit{Staphylococcus} et \textit{Haemophilus} sont faibles dans la plupart des échantillons. Mais pour quelques échantillons, ces genres sont dominants. Les autres genres ont une abondance faible qui varie faiblement. Elles ne sont jamais retrouvé comme dominants.


\begin{figure}
\begin{center}
\includegraphics[scale=0.5]{img/all_table.png}\hfill
\end{center}
\caption{Nombre de reads et prévalence parmis les 188 échantillons.\\ nombre de reads - pourcentage de reads - abondance moyenne - ecart-type - abondance minimum - abondance maximum - prévalence (phrase?)}
\label{alltable}
\end{figure}


\begin{figure}[t]
\begin{center}
\includegraphics[scale=0.5]{img/core.png}\hfill
\end{center}
\caption{Distribution du core microbiota dans l'ensemble des échantillons}
\label{core}
\end{figure}



\begin{figure}[t]
\begin{center}
\includegraphics[scale=0.5]{img/variability.png}\hfill
\end{center}
\caption{Variation du core microbiota dans l'ensemble des échantillons}
\label{violon}
\end{figure}

\subsubsection{Évolution de l'alpha diversité}
Les figures \ref{alphaObs},\ref{alphaChao1} et \ref{alphaShannon} montrent l’évolution des diversités alpha par patient au cours du temps en utilisant les indices Chao1, Observed et Shannon. \\
La richesse (Observed , Chao1) par patient varie de 10 à 40 genres bactériens. La richesse du patient 256 est faible. Celle du patient 26 élevée.
Certains patients ont des richesses stables au court du temps. Les patients 25,26,74 conservent leurs richesses sur plus 5 prélèvements successifs. Les patients 20 et 69 présentent une stabilité globale entrecoupée par des pertes de biodiversité. La richesse du patient 8 diminue progressivement. Les patients 23, 232 et 248 ont des fluctuations plus chaotiques.
L'indice de Shannon montre que pour un richesse constante dans le temps, l'équitabilité est différente. Par exemple la richesse du patient 223 est stable sur le graphique \ref{alphaObs} mais son indice de shannon varie sur le graphique \ref{alphaShannon} témoignant d'une distribution différente des ses bactéries. Seul le patient 26 semble conserver à la fois une richesse et une équitabilité stables au court du temps.

\subsubsection{Évolution des abondances}
Les figures \ref{plotabundancegenre}, \ref{plotabundancephylum} et \ref{plotabundancecurve} montrent l’évolution des abondances au cours du temps pour chaque patient. Ces graphiques nous permettent d’interpréter plus finement les graphiques d’alpha diversité précédente.
Par exemple la faible richesse du patient 8 est liée à une dominance de \textit{Stenotrophomonas} sur les 4 échantillons.
La perte de diversité du 3ème prélèvement du patient 223 est causée par l'apparition de \textit{Neisseria} qui devient dominant.
Le patient 3 montre une diminution progressive \textit{d’Haemophilus} parallèlement à une augmentation de \textit{Streptococcus}. \\
Le patient 256 présente une dominance à \textit{Sténotrophomonas} sur l'ensemble de ses échantillons. \textit{Achromobacter} est dominant dans le premier échantillon du patient 211.
D’autres montrent une phénomène de résilience. Par exemple, le patient 223 récupère un microbiote identique après une colonisation complète à \textit{Haemophilus}. \\
Dans l’ensemble, il existe deux types de population bactérienne: une population de bactéries constantes et minoritaires(\textit{Fusobacterium}, \textit{Granulicatella}, \textit{Gemella}, \textit{Veiillonella}, \textit{Parvimonas},\textit{Leptotrichia},\textit{Oribacterium}, \textit{Capnoctophaga},\textit{Catonella}); Une autre population de bactéries très fluctuantes dans le temps jouant alternativement le rôle de genre dominants ou au contraire du genre absent. (\textit{Haemophilus}, \textit{Streptococcus}, \textit{Neissera} et \textit{Prevotella}).
La figure \ref{evolution43} illustre ces propos en montrant l'évolution des abondances du patient 43.
La figure du patient \ref{evolution54} est mise également à titre exemple, car elle montre l’apparition de \textit{Pseudomonas aeruginosa}. \\
MATERIEL ET METHODS = EXPLIQUER LA PCA 

\begin{figure}
\begin{center}
\includegraphics[scale=0.60]{img/curve_043.png}\hfill
\end{center}
\caption{Evolution des abondances pour le patient 43. Notez la population stable et la population fluctuante}
\label{evolution43}
\end{figure}

\begin{figure}
\begin{center}
\includegraphics[scale=0.60]{img/curve_054.png}\hfill
\end{center}
\caption{Evolution des abondances pour le patient 54. Notez l'apparition de \textit{Pseudomonas} dans le dernier échantillon}
\label{evolution54}
\end{figure}

\subsubsection{Beta Diversité}
La bêta diversité sur l’ensemble des échantillons a été réalisé par une méthode d’ordination de type PCA en utilisant les distances de Bray-Curtis ( Figure \ref{ordination},\ref{ordination2} et \ref{pcoa} ).
2 axes principaux expliquent respectivement 28.5\% et 17.4\% de la variabilité.
Certains échantillons d’un même patient sont très proches sur le graphique d’ordination. Par exemple le patient 69 et 003 ont des échantillons dont les points se confondent.
Aucun des analyses tenant compte des paramètres biocliniquescolligés pendant l'étude MUCOBIOME comme âge, le sexe, le poids, la prise d'antibiotique et le type de mutation du gène \textit{CFTR} n’a mis en évidence des groupes distincts de microbiote. La variabilité est expliqué principalement par la dominance des genres Neisseria, Streptococcus et Haemophilus comme le montre la figure \ref{ordination2}.

\begin{figure}
\begin{center}
\includegraphics[scale=0.60]{img/oordination_new.png}\hfill
\end{center}
\caption{Analyse en coordonnée principale sur les 188 échantillons en utilisant les distance de Bray-curtis. Chaque point est un échantillon labelisé par l'identifiant du patient. A gauche les échantillons Free. A droite les échantillons Never. }
\label{ordination}
\end{figure}

\begin{figure}
\begin{center}
\includegraphics[scale=0.40]{img/Capture.png}\hfill
\end{center}
\caption{Analyse multivarié ( PCA + Bray curtis) sur les 188 échantillons. Chaque point est un échantillon labelisé par l'identifiant du patient.}
\label{ordination2}
\end{figure}


\subsection{Résultats analytiques}
\subsubsection{Corrélation entre les genres bactériens}
La figure \ref{correlation} montre les corrélations linéaires réalisées entre les genres du core microbiota. La corrélation la plus forte est entre \textit{Prevotella} et \textit{Veillonella} avec un coefficient de Pearson à 0.60.  \textit{Streptococcus} et \textit{Haemophilus} evoluent dans le sens inverse avec un coefficient à -0.41. 
 

\begin{figure}
\begin{center}
\includegraphics[scale=0.50]{img/small_correlation.png}\hfill
\end{center}
\caption{Corrélation des abondances entre les genres du core microbiota}
\label{correlation}
\end{figure}

\subsubsection{Comparaison entre échantillons Free et Never}
La figure \ref{compare} compare les indices de Shannon entre les échantillons Free et Never. Le T-test / ANOVA n'a pas montré de différence significative. (p-value > 0.05). \\
La figure  \ref{pcoa} est une analyse multivarié discriminant les groupes Free et Never. Les échantillons Never ont plus de différences entre eux que les échantillons Free qui semblent converger.  Les deux groupes sont significativement différents p-value < 0.001.

\begin{figure}
\begin{center}
\includegraphics[scale=0.70]{img/compare.png}\hfill
\end{center}
\caption{Comparaison des diversités de Shannon entre les échantillons Free et Never}
\label{compare}
\end{figure}


\begin{figure}[t]
\begin{center}
\includegraphics[scale=0.50]{img/pcoa.png}\hfill
\end{center}
\caption{Permanova entre les échantillons Free et Never sur une PCoA en utilisant la distance de Bray-Curtis}
\label{pcoa}
\end{figure}

\newpage
\section{Discussion}
\subsection{Pré-analytique}
\subsubsection{Séquençage}
La version 3 du kit MiSeq Reagent Kits permet de produire des reads plus long ( 300pb ) que dans sa version précedente ( 250pb ). Ceci afin d'augmenter la précision de l'assignement taxonomique ( Figure \ref{rnasens}). En contrepartie la qualité du séquençage est médiocre aux extremités avec des scores qui chutent en dessous de 20. À cause de cela, seulement 43\% de reads sont exploitable. 
C'est un problème connu lié à la chimie du kit qu'illumina n'a toujours pas corrigé. Cependant en raison de la richesse modérée du microbiote pulmonaire, la profondeur de séquençage reste suffisante comme l'attestent les courbes de raréfaction. En utilisant des amorces ciblant d'autres régions, le chevauchement des reads pairés pourrait être plus important et éviterait ainsi de perdre des reads. 

\begin{figure}[t]
\begin{center}
\includegraphics[scale=3]{img/zam.jpg}\hfill
\end{center}
\caption{Sensibilité de détection de la strategie 16S en fonction de la longueur des reads. ( Wang et al. Naïve Bayesian Classifier for Rapid Assignment of rRNA Sequences into the New Bacterial Taxonomy ) }
\label{rnasens}
\end{figure}

\subsubsection{Pipeline}
La stratégie close-reference utilisée dans le pipeline, a permis d'assigner le genre à plus 99\% des reads analysable. La majorité des genres bactéries du microbiote respiratoire est donc connue de la base de données \textit{greengene}.
En revanche, la résolution taxonomique n 'atteint pas le rang de l'espèce. Il a en effet déjà été montré que les régions hypervariables prises isolément ne sont pas assez discriminantes \cite{Yang2016}. Les analyses en métagénomiques apportent ici un grand avantage en séquençant l'ensemble de l'ARN 16S \cite{Ranjan2016}.
D'autre part, l'assignation taxonomique dépend d'une similarité fixé arbitrairement à 97\%. D'autres techniques de clusterisation permettent de s'affranchir de ce seuil. Il s'agit par exemple de la \textit{Maximum entropy clusterisation} \cite{Bobadilla2015}. 
Enfin, la phylogénie des bactéries n'a pas été prise en compte dans l'étude. Elle permettrait de prendre de mesurer une distance phylogénétique entre les OTUs et comparer les échantillons. Un microbiote contenant 3 espèces proches n'est pas la même chose qu'un microbiote avec 3 espèces très éloignées. La distance UniFrac\ref{Lozupone2005} serait alors utilisée afin de remplacer la distance de Bray-Curtis. 

\subsection{Analytique}
\subsubsection{Augmentation des Bactéroidetes}
Nos résultats chez les patients mucoviscidoses montrent une augmentation des Protéobacteria aux dépens des Bactéroidetes. En effet chez le sujet sain, les propositions sont plutôt de 10\% de Protéobactéria et une proportion égale de 40\% pour les Bactéoridete et de Firmicutes. 
Ceci corrobore les résultats de [@][@] retrouvant ces mêmes proportions dans la Mucoviscidose, l’asthme et la BPCO. Cet excès s’explique principalement par la dominance d’Haemophilus et de Neisseria. 
\subsubsection{Infection et inflammation}
Les principaux pathogènes associés à la mucoviscidose sont tous retrouvés avec une tendance à dominer les autres. Pourtant, aucun patient dans l'étude n'était en exacerbation. Ceci évoque l'absence de corrélation entre inflammation et colonisation proposées par (truc). 

\subsubsection{Présence de Pseudomonas}
Malgré la sortie de l’étude de tous les patients positifs en culture,  P.aeruginosa est retrouvé dans 53 échantillons. Ceci confirme la faible sensibilité de la culture et montre l’intérêt d’autre méthode de diagnostic comme la PCR\cite{LeGall}.

\subsubsection{Présence d'anaérobie}
De fortes abondances en anaérobies  sont également retrouvées avec notamment Prevotella et Veillonella. La corrélation forte  entre ces deux genres a été retrouvée dans l’étude de [@]. La signification clinique de leurs présences est encore incertaine \cite{Tunney2008}. 

\subsubsection{Dynamique du microbiote}
Les résultats longitudinaux nous montrent la dynamique et la complexité du microbiote respiratoire. 
Deux populations se distinguent. Une population stable que l'on peut qualifier de commensale et une population fluctuante et dominante. Ceci évoque le modèle attack-climax \cite{Conrad2013}. Les infections répétées des bactéries (population d'attaque) feraient le lit à l'émergence des bactéries déjà présentes ( population Vivax) comme \textit{Pseudomonas aeruginosa}.  

\subsubsection{Patient Free et Never}
Aucune des données récoltées n’a mis en évidence un groupe particulier de microbiote. L’âge n’a pas été corrélé à la présence d’haemophilus, ni à l’augmentation de diversité évoquée dans plusieurs papiers. Probablement, car les sujets de l'étude avaient des âges proches. La comparaison d'alpha diversité entre les patients Free et Never n'as pas relevé de différence significative. Il n'y a pas plus d'espèces dans l'une ou l'autre des catégories. 
L'analyse multivariée s'explique principalement par les grandes fluctuations d'abondance de Streptococcus, Neisseria et Haemophilus. Elle montre également qu'il y a plus de microbiotes différents chez les patients Free que Nevers. Probablement causé par l'impacte du traitement anti-pyo.  \\
Il semble donc ressortir de l'étude que le microbiote dépend avant tout du patient. La concordance de certains échantillons pour un même patient sur les graphiques d'ordination va dans ce sens.  [citation papier]

\subsection{Limite de l'étude}
Plusieurs points sont à prendre en considération pour critiquer la justesse des résultats. Premièrement, les ADN séquencés proviennent d'expectoration bronchique ayant traversé les voies aériennes supérieures. Une contamination est toujours possible. Elle est cependant assez réduite, car nous avons pris en considération le score cytologique de Bartlett.  
Une différence entre deux échantillons d'un même patient pourrait s'expliquer par l'hétérogénéité du poumon. Il a cependant été montré [@] sur des prélèvements in situ que le microbiote pulmonaire est homogène.  
Ensuite la stratégie 16S n'est pas parfaite. La détection varie en fonction des amorces utilisées (figure). Et d'autres espèces ne sont tout simplement pas amplifier par cette stratégie[@]. De plus, cette stratégie ne fait pas la distinction entre les ADNs de bactéries vivantes ou morte.  

\begin{figure}
\begin{center}
\includegraphics[scale=0.70]{img/enfin_barplot_genus_norm.png}\hfill
\end{center}
\caption{Evolution des abondances par genre}
\label{plotabundancegenre}
\end{figure}

\begin{figure}
\begin{center}
\includegraphics[scale=0.70]{img/enfin_barplot_phylum_norm.png}\hfill
\end{center}
\caption{Evolution des abondances par phylum}
\label{plotabundancephylum}
\end{figure}


\begin{figure}
\begin{center}
\includegraphics[scale=0.50,angle=90]{img/evolution_abundance.png}\hfill
\end{center}
\caption{Evolution des abondances par genre}
\label{plotabundancecurve}
\end{figure}


\begin{figure}
\begin{center}
\includegraphics[scale=0.80]{img/alpha_observed.png}\hfill
\end{center}
\caption{Evolution du nombre d'espèces par patient en fonction du temps}
\label{alphaObs}
\end{figure}


\begin{figure}
\begin{center}
\includegraphics[scale=0.80]{img/alpha_chao1.png}\hfill
\end{center}
\caption{Evolution de l'indice de Chao1 par patient en fonction du temps}
\label{alphaChao1}
\end{figure}

\begin{figure}
\begin{center}
\includegraphics[scale=0.80]{img/alpha_shannon.png}\hfill
\end{center}
\caption{Evolution de l'indice de Shanon par patient en fonction du temps}
\label{alphaShannon}
\end{figure}


\section{Conclusion}




\newpage


\bibliographystyle{plain}
\bibliography{biblio}

\end{document}
