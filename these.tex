\documentclass[12pt,a4paper]{article}
    % for Nature's style citations
\usepackage[super]{natbib}
\usepackage[utf8]{inputenc}
\usepackage[francais]{babel}
\usepackage[T1]{fontenc}
\usepackage{graphicx}
\usepackage{fancyhdr} % Needed to define custom headers/footers
\usepackage{setspace}
\usepackage{listings}
\usepackage{color}
\usepackage[scale=2]{ccicons}
\usepackage{caption}
\usepackage{url}
\usepackage[final]{pdfpages}
\usepackage{caption}
\usepackage{svg}
\usepackage{listings}
\usepackage{eurosym}
\usepackage{wrapfig}
\usepackage{subfig}
\usepackage[margin=2.7cm]{geometry}
    

    %%%%%%%%%%%%%%%%%%%%%%%%
    %%%%% MISE EN PAGE %%%%%
    %%%%%%%%%%%%%%%%%%%%%%%%
    %Interligne à 1.5
    \onehalfspacing





    \pagestyle{fancy} % Enables the custom headers/footers
    \lhead{Mémoire M2 }
    \chead{Université Rennes 1}
    \rhead{Sacha SCHUTZ}
    \lfoot{M2 Bioinformatique et Génomique}
    \rfoot{2015/2016}
     
     
    %%%%%%%%%%
    % MACROS %
    %%%%%%%%%%
    \newcommand{\HRule}{\rule{\linewidth}{0.5mm}} % Defines a new command for the horizontal lines, change thickness here
    \newcommand\nt{nucléotides }
     
    \newcommand{\includefigure}[3] {% label, caption, width ratio
    %ex\includefigure{RkNN}{Graphe d'exemple et R1NN/R2NN associés.}{0.8}
            \begin{center}
         \includegraphics[width=#3\textwidth]{#1}
         \captionof{figure}{#2}
         \label{FIG:#1}
            \end{center}
    }
     


\begin{document}


%%%%%%%%%%%%%%
% TITLE PAGE %
%%%%%%%%%%%%%%
\begin{titlepage}
\center







%       HEADING SECTIONS
\textsc{\LARGE Thèse de médecine}\\[1.5cm]
\textsc{\Large CHU-BREST}\\[0.5cm]
\textsc{\large Université de Brest\\
Brest \
}\\[0.5cm]
 
%       TITLE SECTION
\HRule \\[0.8cm]

{ \huge \bfseries Description du microbiote pulmonaire chez les patients atteints de mucovicidoses}\\[0.4cm]

\HRule \\[1.2cm]
 
%       AUTHOR SECTION
\begin{minipage}{0.4\textwidth}
 \begin{flushleft} \large
     \emph{Auteur:}\\
     Sacha SCHUTZ
 \end{flushleft}
\end{minipage}
~
\begin{minipage}{0.4\textwidth}
 \begin{flushright} \large
     \emph{Responsable:} \\
     Geneviève HERY-ARNAUD
 \end{flushright}
\end{minipage}\\[2cm]
 
{\large \today}\\[8cm] % Date, change the \today to a set date if you want to be precise
 
% LOGO SECTION
\begin{minipage}[c]{0.3\textwidth}
   \includegraphics[width=0.7\textwidth]{img/logo_brest.jpg}\hfill
\end{minipage}
\begin{minipage}[c]{0.3\textwidth}
   \includegraphics[width=0.7\textwidth]{img/ubo.png}
\end{minipage}
%\begin{minipage}[c]{0.3\textwidth}
%        \includegraphics[width=0.7\textwidth]{img/logo_brest.jpg}     
%\end{minipage}
% 
 
 
 
\vfill % Fill the rest of the page with whitespace

\end{titlepage}


 %----------------PAGE ENGAGEMENT ET LICENSE ---------------------------

\newpage

\section*{Engagement de non plagiat}

Je, soussigné Sacha SCHUTZ, interne en biologie moléculaire au CHU de Brest, déclare être pleinement informé que le plagiat de
documents ou de parties de documents publiés sur toute forme de
support, y compris l'internet, constitue une violation des droits
d'auteur ainsi qu'une fraude caractérisée.

En conséquence, je m'engage à citer toutes les sources que j'ai
utilisées pour la rédaction de ce document.

Date : 17/05/2017

\vspace{0.5cm}

Signature : \\

ssh pub key fingerprint : a4:e3:da:87:78:2d:e1:6f:bb:56:5c:d1:72:f5:50:63
\vfill 

\section*{License}

\begin{wrapfigure}{R}{0.3\textwidth}
\includegraphics[scale=0.5]{img/gfdl.png}\hfill
\end{wrapfigure}

Copyright (c) 2015 SCHUTZ Sacha. Permission est autorisée de copier,
distribuer et/ou modifier ce document sous les termes de la Licence de
Documentation Libre GNU, Version 1.2 ou toute version ultérieure publiée
par la Free Software Foundation ; with no Invariant Sections, no
Front-Cover Texts, and no Back-Cover Texts. Une copie de la license est
incluse dans la section intitulée ``GNU Free Documentation License''.

 %----------------PAGE REMERCIEMENT: A FAIRE.. ---------------------------
\thispagestyle{empty} 
\setcounter{page}{0}
\thispagestyle{empty} 

\newpage

\tableofcontents
\newpage


\section{Avant-propos}

Depuis Pasteur, les microbes ont toujours été associé aux maladies. Ils sont les agents nuisibles devant être eradiqué dans le fantasme d’un corps stérile comme signe de bonne santé.
La mise en évidence des bactéries pathogènes allant de la syphilis jusqu’au grande Peste n’a pas aidé ces êtres microscopique à sortir de ce stéréotype. La médecine les ont donc naturellement choisi comme cible privilégiée en développant l’hygiène, les vaccins et les antibiotiques.
Personne ne peut nier que cette triade a permis l’amélioration de notre santé en diminuant la prévalence des maladies infectieuses. Mais la disparitions de nos “vieux amies”[@] ayant co évolué avec nous depuis des milliers d’année, est associé, dans les pays industrialisé, à une recrudescence de nouvelle maladie[@] comme l’asthme, le diabète de type 1 ou encore la maldie de Crohn.[Théorie hyieniste].
Aujourd’hui, les nouvelles méthodes d’exploration de ce monde microscopique, comme le séquençage de l’ARN 16S, permet aux bactéries de redorer leurs blazons.
Un role majeur leurs à d’abord été attribué en botanique par l’étude des sols. Ils jouent en effet le rôle principal dans le cycles de l’azote en permettant à la biomasse d’absorber l’azote athmosphérique. Par la chaine allimentaire, les bactéries sont la seul sources d’azotes permettant de consruire nos protéines et notre ADN. Sans le soleil il n’y aurait pas d’homme, sans les bactéries non plus.
Les bactéries ont ensuite été retrouvé dans tous les environements, y compris les plus inhospitalier. Notemment les archébactéries qui peuvent vivre dans des conditions d’acidité et de chaleurs exceptionnelles. C’est d’ailleurs leurs découverte qui nous a éclairé sur l’origine des eucaryotes[baspage] et qui nous a permis de faier un bon de géant en biologie moléculaire[baspage].
Chez l’homme sain, Les régions autrefois considéré stérile d’après la culture, floisonne maintenant de bactérie. En effet, la majorité des bactéries ne pouvant pousser en cultures, elles ont longtemps été indetectable.
Elles sont retrouvé dans toutes les régions du corps exposés ou elles forment des communautés.
La peau est colonisé par Propionibacterium, Corynebacterium et du Staphylococcus[@]. Le vagin contient les bacille de Döderlein et la bouche princiapelement du Streptococcus[@].
L’intestin est une flore bactérienne dominé par les anéorobie et pouvant representé jusqu’a 2 kg du poids corporel.[@]
En échange de son hospitalité, notre microbiote participe au bon fonctionement de son hôte. Il aide à la digestion en dégradant par exemple les sucres du lait maternelle ne pouvant être digéré par le nouveaux née[@ bifidus]. Il participe à la synthèse de vitamine essentiel (K, B12,B8)[ref]. Il éduque notre système immunitaire et fait barrières à tout nouvelle agent pathogène.
Tout defaillance de ce nouvel organe ou dysbiose, peut s’accompagner d’un problème de santé. La liste des maladies associés à une dysbiose est longue. On retrouve la maladie de Crohn, la maladie coeliaque, le cancer de l’intestin, le syndrome du colon irritable, l’obésité, le diabète de type 1, l’asthme, l’eczéma, la sclérose en plaque, la polyarthrite rhumatoïde, la maladie d’alzheimer et même l’autisme.
La colite à clostridium difficile qui survient suite à un traitement antibiotique est une application directe des théories microbiotique. En effet, ce traitement détruit aussi bien les pathogènes que la flore intestionale. Un des traitements proposé est la transplantation fécale qui consiste à réintroduire un microbiote au patient.
[
Notion de bactérie specifique à notre génome.
qque famille seulement parmis une centaine.
]
Le microbiote amène donc à reconsidérer notre individualité. Nous ne sommes pas qu’un eucaryotes multicellulaire composé d’un unique génome. Mais un écosystème ou cellules eucaryotes et microbienne vivent en symbiose mutualiste. Les dernières études estiment que pour chaque individus il y a environs 30 billions de cellules humaines pour 39 billions de cellules microbiennes[@]. De plus, en associant les gènes bactériens, le génome d’un individu passe de 23 000 gènes à 3,3 millions[wiki] de gènes avec toutes la complexités des interactions que cela engendre.[@] Les scientifiques ont donnée le nom d’holobionte à cette entité vivante hétérogène. Pour pousser le vis, l’ensemble du génome humain et microbiens est appelé hologénome. Ce dernier est pour certain la cible de la sélection naturelle.
Il faut tout fois rester prudent quant au role donné aux microbiote et éviter de tomber dans l’excès. Nombreux sont les publications scientifiques qui se contredisent et qui ne montre que des corrélations, sans rapport direct de causalité. Cette excès de publication à même conduit à la création d’un hashtag humouristique sur twitter :GutMicrobiomeAndRandomSomething[https://microbiomedigest.com/2015/07/01/microbiome-and-random-thing/] ou les gens publiait les corrélations les plus absurde.
En effet, il n’y pas que l’hygiène qui a changer dans nos sociétés. D’autre facteur, comme notre mode de vie sedentaire ou notre allimentation, peuvent tout aussi bien être impliqué dans la survenue de maladie.
Les études sur le microbiotes necessitent d’être réalisé à plus grande echelle avec plus de patient et avec un suivi à long terme important. Les nouvelles technologies de séquençage haut débit vont dans ce sens en permettant d’ammaser des quantités de donnée phénoménale limité seulement par les capacités de calculs.
Il n’y a pas de honte à dire qu’aujourd’hui, nous ne connaissons pas grand chose au sujet du microbiote humain. C’et une science naissante et seul le futur nous dira si il s’agit d’un effet de mode ou d’une révolution.
Pour ma part, au regard de la théorie de l’Evolution, je parrierai dessus. Car ce sont bien des anciennes bactéries, qui permette à l’ensemble de nos cellules de réspirer et que nous appelons maintenant mitochondries.
[Pas de probiotic miracle]
[PTetre qu’en regardant autre chose que le Gut, on trouvera un truc]
Les mécanismes d’aide du microbiote :

Effet de barrière
stimule le SI

\newpage

\section{Définition}

\begin{description}
  \item[Le microbiote] est l’ensemble des micro-organismes (bactéries, levures, champignons, virus) vivant dans un environnement donné.
  
\item[Le microbiome] s’emploi selon deux définitions. En français, le microbiome est l'environnement qui héberge le microbiote. Dans sa définition angle-saxonne, le microbiome fait référence à l’ensemble des génomes microbien contenu dans un environnement. 
De façon général le microbiome est associé aux génomes bactériens. Les termes de Virome et de Mycobiome sont utilisé pour les génomes viraux et myoctiques. 

\item[La biocénose] est le terme écologique dans un sens large désignant l'ensemble ds organismes vivants dans un environnements appelé \textbf{Biotope}. Biocénose et biotope forme ensemble un \textbf{écosystème}.

\item[Une symbiose] est une association durable entre deux organismes. Leurs relations peuvent être mutualistes, parasitaire ou commensale.

\item[La métagénomique] est une méthode d’étude du contenu en ADN présent dans un milieu grâce au technique de séquençage haut débit. Contrairement à la génomique qui s’intéresse au génome d’un individu, la métagénomique s’intéresse aux génomes d’une population d’individu.
Dans son sens stricte, la méta-génomique correspond à l’étude de l’ensemble des séquences d’ADN. L’analyse d’un seul gène, comme celui de l’ARN 16s est associé à tord au terme métagénomique, mais son usage reste courrant. On lui préférera le terme de \textbf{metagénétique}

\item[Un read] est un terme bioinformatique désignant une séquence d’ADN issue d’un séquençage haut débit. Selon les technologies, les reads varient entre 150 et 300 paire de bases.

\item[Un OTU] (\textit{Operational taxonomic Init}), est un terme utilisé en phylogénie, désignant un groupe d’individu proche et fait souvent réference à l’espèce dans la classification de Linée.
En microbiologie, un OTU fait référence à un groupe d’individu ayant une similarité dans leurs séquence d'ARN 16s supérieur à 97\%

\item[L'Abondance] absolu est le nombre de séquences d’ADN d’un OTU retrouvé dans un échantillon. 
L’abondance relative est le pourcentage en séquences d'ADN d'un OTU retrouvé dans un échantillon. Ce dernier indicateur permet de rendre les échantillons comparable entre eux.


\item[La table des OTU] correspond à un tableau à double entrée contenant l’abondance d’un OTU pour chaque échantillon. Dans le tableau suivant, l'échantillon 1 contient 68\% de l'OTU 1.

\begin{figure}
\begin{center}
\begin{tabular}{|l|c|c|c|c}
  \hline
   & échantillon 1 & échantillon 2 & échantillon 3  \\
  \hline
  OTU 1 & 68\% & 12\% & 25\% \\
  OTU 2 & 40\% & 24\% & 25\% \\
  OTU 3 & 28\% & 64\% & 50\% \\

  \hline
\end{tabular}
\end{center}
\caption{La table des OTUs}
\end{figure}

\item[La diversité alpha] est une mesure de biodiversité au sein d’un échantillon. Elle correspond donc à l’étude d’une colonne dans la table des OTUs. Plusieurs indicateurs de diversité alpha.

\item[La diversité beta] est une analyse descriptive de la biodiversité entre plusieurs échantillons. Elle correspond à l’étude de l’ensemble de la table des OTUs. L’approche la plus courante est de réalisé une analyse multivarié par des méthodes d’ordination. Il s’agit de représenter un graphique à N dimensions, impossible à dessiner, en le projetant dans un espace à une ou deux dimensions.


\item[La richesse] est le nombre d’espèce présent dans un échantillon. Les deux échantillons suivant ont la même richesse de 2. 

échantillon 1  : 4 Streptoccus , 4 Escherichia  \\ 
échantillon 2 : 432 Streptoccus, 12 Escherichia 

\item[L'uniformité] indique si les espèces d’un échantillon sont répartis uniformément.
L'uniformité du premier échantillon est plus grand que le second

échantillon 1  : 50 Streptoccus , 50 Escherichia  \\ 
échantillon 2 : 432 Streptoccus, 12 Escherichia 


\item[L'indice Chao1] est une estimation de la richesse réel (in vivo) par rapport à la richesse observé (in vitro). Cette indice part du principe que si l’échantillon contient beaucoup de singletons ( OTU détecté une seul fois), il est fort probable que la richesse réel soit plus grande que la richesse de l’échantillon. La formule est la suivante.
\begin{equation}
\end{equation}


\item[L'indice de Shannon] est un indicateur évaluant à la fois la richesse et l’uniformité dans un échantillon. Il se calcul de la même façon que l’entropie de shannon.

\begin{equation}
\end{equation}

\item[L'indice de Shannon] est un indicateur évaluant la probabilité que deux individus sélectionnées aléatoirement dans un échantillons donnée soient de la même espèces. La formule est la suivante.

\item[La courbe de rarefaction] est utilisé pour determiner si la profondeur de séquençage est suffisante pour caracterisé la diversité d’un échantillon.
Pour génerer cette courbe, des groupes de reads de taille croissante (1…n) sont tiré aléatoirement sans remise. Pour chaque groupe en absisse le nombre d’OTU correspondant est reporté sur l’axe Y.
Une courbe s’applatissant indique qu’une profondeur de séquençage plus grande, n’apporterai pas plus d’information. \citep{Dickson2014} and \citep{Beck}

\end{description}

\setcounter{page}{1}

\section{Introduction}
\subsection{La mucovicidose}
\subsubsection{Une maladie génétique}
La mucovicidose est une maladie génétique autosomique recessive grave touchant en france 1 naissance sur 5400 [@]. La bretagne est la région la plus touché avec une prévalence de 1/3000[@].
La loi de Hardy Weinberg estime qu’en bretagne 1 patient sur 25 est porteur de la mutation à l’état hétérozygotes[@Heterozygote advantage]. Cette haute prévalence s’explique probablement par un effet fondateur et d’un avantage séléctif pour les individus porteur de l’allèle muté. (Plusieurs hypothèses ont été proposé, notamment lors des grandes épidémies de choléra en diminuant les pertes hydriques [@]. D’autre hypothèse suggère une améliration du fitness chez les individus atteinds de tuberculose. Et d’autres qu’il s’agit d’un exemple de pleiotropie antagoniste.[@])
Le gène CFTR impacté se situe sur le chromosome 7 en position []. Il est constitué de 27 exons sur 250,188 [@] paire de bases. Il code pour une canneau chlore AMp dépendant permettant les échanges des ions chlorures au niveau des membranes cellulaire.[@]. Il est egalement impliqué dans le transport du thiocynate (SCN-) et des bicarbonate (HCO3-).[@].
On dénombre à ce jour, 2017 mutations impliqué dans la mucovicidoses[@]. La perte d’une phénylanine en position 508 par deletion du codon (delta F508 (new nom)) est responsable à elle seul de 80% des mucovicidoses.
Les mutation sont responsable soit d’une protéine défectueuse ou d’une absence de canaux sur les membranes cellulaire.
D’un point de vue clinique, la mutation est responsable chez les patients d’une insuffisance pancreatique exocrine et d’une infertilité par disparition des cannaux déferent. Des signes digestif, hépatique et articulaire sont egalement retrouvé.
Mais l’atteinte pulmonaire est la plus bruillante. En effet au niveau de l’epithélium broncho-pulmonaire, l’absence d’un CFTR fonctionnelle est à l’origine d’une desyhdration du mucus le rendant plus visceux et empéchant les cils bronchiques de jouer leurs role.[@]
La forte prévalence de la maladie nécessite de réaliser un dépistage précoce chez tous les nouveaux née (test de Gutri) afin d’adapter au plus tôt la prise en charge. Seul le test à la sueur permet de poser le diagnostic. Des test de dépistage prénatal basé sur l’ADN circulant sont actuellement à l’étude.
Il n’y a pas de traitement curateur à l’heure actuelle. Le traitement repose avant tout sur une prise en charge réspiratoire (kinésithérapie, dornase, antibiothérapie).
Les thérapies génétiques sont encore à l’étude[has]
L’Ivacaftor est le seul traitement à ce jour qui agit directement sur le CFTR. Maise concerne uniquement certaine mutation de classe 3 dont la plus courrante G551D.[has]
La greffe pulmonaire est le derniere recours.

\subsubsection{Une maladie infectieuse}


[ Princiaple cause de mobidity / mortalité dans la muco [Nixon et all. 2001]]
L’atteinte pulmonaire est caractérisé par des infections successive associé à une réaction inflammatoire qui dégrade progressivement la fonction réspiratoire.
Plusieurs pathogènes sont décrit. Chez les jeunes enfant, Haemophilus influenza et Staphilococcus dorée sont majoritairement retrouvé. Burkolderia cepacé et stenotrophomonas maltophilia egalement retrouvé chez le sujet plus agé.
Mais c’est Pseudomonas Aeruginosa qui caractérise l’atteinte pulmonaire dans la mucovicidose en marquer un tournant décisive dans l’évolution de la maladie. Ce bacile aérobie strictes, est un germe de l’environement rarement retrouvé chez les patients sains. En revanche il est retrouvé chez 60% des patients muco jeune, et plus de 90% des patients adulte[@]. La primocolonisation est difficilement detectable, mais semble avoir lieu tôt dans l’enfance[@]. Il y a ensuite une phase de latence, variable entre individus, marqué par des épisodes d’exacerbation. A ce moment l’éradiction[*?] par des antibiotiques reste possible
Puis survient le passage à la chronicité. Pseudomonas s’adapte à son milieu et s’installe à long terme. Il perd certaine caractère de virulence, mais devient résistant au antibiotiques. Son phénotype change. Il devient mucoïde en sécretant un film d’alginate qui le protège du système immunitaire. Les mécanismes sous jacent à cette transformation sont ingenieux. La forte densité en bactérie est résponsable d’activation de certain gène amenant au phénotype mucoïde par un processus appelé quorum sensing. un processus dans lequel chaque bactéries communique avec ses voisins via des signaux. ( On peut comparé ce processus à un système multiagent. comme un banc de poisson ou le comportement global dépend du comportement d’un individu).
Les génomes de pseudomonas aeruginosa deviennent aussi hypermutable. Et par conséquence présente une plus grande diversité génétique au regard de la selection naturelle. Pour un biologiste de l’evolution, il s’agit d’un cas d’evolvabilité.
A ce stade le traitement antibiotique n’est plus curatif et l’evolution tent inexorablement vers un declin de la fonction réspiratoire.
On ne sait pas aujourd’hui exactement pourquoi Pseudomonas aerugiona s’installe preferentiellement chez les patients muco. Plusieurs hypothèse ont été posé.

La dysfonction cilliaire empeche les les Pseudo d’etre viré
L’hypersalinité du film muqueus desactive les peptides antimicrobien
Le CFTR est un recepteur de Pyo pour les internalisé et les viré
L’inflammation de de l’eopthelium augmente les metabolite qui permette de se developper.
Alanine et lactta sont une source de carbonne pour le Pyo. [89 Host microorganisme]
D’un point de vue clinique, l’approche est préventif, visant à éliminer le pyo dès qu’il est detécté en culture. Une surveillance rapproché des patients muco avec un prélevement mensuel ou bimensuel est préconsié. La culture étant peu sensible, d’autre approche peuvent être utilisé pour détécter le pyo . La détéction des anticorps anti-pyocianique par des méthodes élisa a montrer peu de …[]
La PCR ciblé s’est montré plus sensible et spécifique que la culture.
En pratique,la colonisation chronique est défini comme étant 3 résultat positifs successifs au cours d’un suivi mensui ou bi mensuel.
Une autre classification a été defini par Lee [@]. Groupe Free et Never.

\subsection{Le microbiote pulmonaire}

Bien qu’il soit en contacte avec le milieu exterieur, l’arbre réspiratoire (comprenant la traché, les bronches et les alvéole) a longtemps été considéré comme stérile avec les méthodes de culture classique. Il a fallu attendre l’avenement du séquençage haut débit pour les mettres en évidence. pulmonaire[ref Host-microorganism 1-3].
Le microbiote pulmonaire, est beaucoup moins abondante que la flore digestif. Il est constituté d’une flore dynamique provenant essentiellement de l’air ambiant mais aussi du tube digestif via des microaspirations.[@]
Le microbiote pulmonaire est dominée par le phylum des Firmicutes ( Streotpococcus) et des Bacteroidetes (Prevotella).[pie] Les genres retrouvé majoritairement sont Streptococcus, Prevotella, Fusobacteria, VBeillonella, Haemophilus, Neisseria et Porphyromonas.
l’arbre réspiratoire étant en continuité direct avec les voies aérienne superieur, certain genre bactérien sont commun, comme Streptococcus, staph, Haemophilus et Moraxella. Tandis que d’autre genre comme corynebacterium et Dolosigranulum ne sont retrouvé qu’au niveau des voies aérienne superieur.

Le microbiote est variable dans l’espace et le temps.
Dans l’espace, suivant les régions préelevé du poumon. [differente zone, foyer inflammatoire]. Le poumon presente des régions variable en Oxygene et en Ph . Et certaine bactérie sont anaeobie stricte, et pas d’autre?

Plusieurs études suggère une différence de microbiote entre patient sains et avec une atteinte réspiratoire chronique comme l’asthme, la BPCO et la mucovicidose [ Huang et all 2010].

Plusieurs études suggère que certain microbiote sont associé à des pathologies comme l’asthme, la BPCO ou la mucovicidose. Le mécanisme sous jacent, théorie hygieniste, stipule que les bactéries stimule le Systeme immunitaire.

propre a chaques individus
variable dans le temps
variable selon les localisation
résilience des population
les résident peramanent et les attack
Liens avec les maladies :
– hygeniste

\subsection{Exploration du microbiote pulmonaire}

(http://bacterioweb.univ-fcomte.fr/bibliotheque/remic/08-Bronc.pdf) ==> A LIRE COOURS BACTERIO
Le microbiote pulmonaire est explorer par le séquençage des ADNs bactériens présent dans un prélevement broncho-pulmonaire. Toutes les méthodes de recceuils sont possible, mais les prélevement protégé sont recommandé afin d’éviter une contamination par les voies superieurs.
L’ADN est alors extrait …
2 strategies de séquençages peuvent être employé:
Le stratégie shotgun séquence l’ensemble du contenu en ADN présent dans l’échantillons après une fragmentation des ADNs. Les algorithmes bioniformatiques consistent ensuite à d’abord supprimer les ADNs humains. Puis à aligner ensemble les ADNs restant afin de reconstruire les génomes bactériens.

consiste à amplifer une région d’ADN assez disscriminante pour identifier une espèces. Cette amplicon est ensuite séquencé. Les algorithmes bioinformatiques assigne alors à chaque reads son taxon approprié.

La première stratégie est plus complexe d’un point de vue calculatoire que la seconde mais est beaucoup plus informatif. En effet en raison des transfert génétique horizontaux, la fonction du microbiote est plus lié aux gènes présents qu’au espèces présentes.
La stratégie 16s reste une bonne alternative pour décrire les populations bactériennes.

\section{Materiel et Méthodes}
\subsection{Recueil des données}

47 patients atteind de mucovicidoses ont été suivit sur 3 ans ( 2008-2011) dans une étude prospective multicentrique (Nantes,Brest,Roskoff) appelé mucobiome.
La figure () montre la date des prélevements .
La CPP VI-Ouest et le commité d’éthique du CHRU de Brest ont approuvé le protocole. Tous les patients ( ou les parents pour les mineurs) ont signé un consentement éclairé. Le protocole à fait l’objet d’une déclaration de biocollection à l’ARS et au MESR (n DC-2008-214).
Dans le suivi planifié des patients, les crachats lavés des patients ont été receuillies lors des séances de kinésithérapie réspiratoire tous les 3 mois, suivant le calendrier des recommandations officielles. En pratique, sur l’ensemble de la cohorte suivie, l’intervalle median entre 2 consultations a été de 3.4 mois.
Les patients devaient avoir un genotypage CFTR et un test à la sueur positif. Les transplanté ont été exclus de l’études.
Une culture positive à Pyo était un critère de non-inclusion. Si pendant l’étude, une culture revenait positif à Pyo, le patient était sortie de l’étude pour être réinclus 1 ans après en l’absence de colonisation chronique à Pyo. 15 patients ont été ainsi réinclus.
Chaque patient était classé dans la categorie Free ou Never (Lee et all 2003). D’autres données ont été également recceuilli ( tableau).
Au total , 188 échantillons ont été recueilli, soit en moyenne 4 échantillons par patients.
Pour chaque échantillon, une culture a été réalisé en suivant les procédure standard [ref]. Une qPCR ciblant le Pyo a également été réalisé en combiant les marqueurs gyrB/ecfX designé au laboratoire[ref].

\subsection{Extraction de l’ADN}

Les échantillons ont été liquéfié avec du Dithiotrétiol . Les protéines ont été degrédé avec une Proteine kinase.
Les parois bactériennes ont été fragmenté par sonication. (DTTpar sonication (Elamsonic S10, Singen, Germany). Après 10 min de centrifugation, L’ADN a été extrait à partir du culot via QUIAamp DNA Minikit ( Quagen).
Les extraits d’ADN ont été envoyé pour séquençage via un prestataire GATC.

\subsection{Séquençage}

La séquençage d’une région de l’ARN 16S a été réalisé sur Illumina MiSeq.
La librarie a été crée en amplifant la région V3-V5 de l’ARN 16s à l’aide du couple d’amorces A B et du kit MiSeq Reagent Kits v3.
Ce dernier permet de produire des reads pairé chevauchant de 300 pb chacun. Environ 25 millions de reads sont produit par run.
En multiplexant à l’aide de 94 index, 2 runs ont permis de séquencer les 188 échantillons.
Au final 188 x 2 fichiers fastq ont été généré à l’issue du séquençage.

\subsection{Analyse bioinformatique}

l’analyse des 188x2 fichiers fasq a été réalisé grâce un pipeline bioinformatique appelé “mucobiome” conçu et tester dans le cadre de cette thèse. Par rapport aux autres logiciel comme QIIME ou MOTHUR, le pipeline mucobiome est spécialisé dans l’analyse des données 16s et dépend de très peu de dépendance applicatives. Mais il est surtout beaucoup plus rapide, ceci en raison d’un très haut niveau de parallélisation permis grâce un Snakemake. Snakemake est un programme informatique permettant de distribuer les différentes operations de calcul du pipeline sur plusieurs processeurs en même temps. Pour illustrer ces propos, imaginons qu’un pipeline est une recette de cuisine, composé de plusieurs étapes successif. Sans parallélisation, un cuisinier doit attendre que chaque étape se termine avant de passer à la suivante. Faire fondre le beurre, puis dans un second temps battre les oeufs en neige ( Exécution synchrone). En parallélisant, le cuisinier peut faire plusieurs étapes en même temps. Battre les oeufs pendant que le beurre fondent. ( Exécution asynchrone). Maintenant, si l’on demande à 40 cuisiniers ( processeurs) de faire 188 gateaux, l’organisation des tâches devient complexe si l’on veut distribuer toutes les tâches de façon à maximiser les performances. C’est le problème que résout Snakemake en réalisant ce qu’on appelle un DAG ( Direct Acyclic Graph) .
Le pipeline mucobiome prend comme entré, l’ensemble des fichiers brut de séquençage pour produire un seul fichier unique contenant la table des OTU et la taxonomie . L’ensemble des étapes est défini ci-dessous .

\subsubsection{Pré-Traitement}

\subsubsection{Verification des qualités}

\subsubsection{Fusions des reads}

Les données bruts provenant d’un séquenceur sont des fichiers FastQ ( annexe ). Ces fichiers contiennent les séquences nucléotidique et les scores de qualités de l’ensemble des reads lu par le séquenceur. La stratégie de séquençage étant paired-end, pour chaque échantillon séquencé, deux fichiers fastq sont fourni. L’un correspond à la séquence lu dans le sens forward et l’autre lu dans le sens reverse.
La première étape du pipeline consiste donc à fusionner ces deux fichiers . c’est à dire fusionner les reads deux à deux afin de produire un plus long reads de 500 pb en moyenne. Cette plus longue séquence correspond à la région V3-V5 de l’ARN 16S.
2 algorithmes ont été utilisé pour la fusion des reads, et sont mis à disposition de l’utilisateur.

Flash

VSearch

\subsubsection{Trimming des reads}

Afin d’assurer un alignement parfait, les séquences ne contenant pas à leurs extremités les amorces V3-V4 ont été supprimées ou ajuster.
L’algorithme utilisé est celui de cutadapts. Cette algorithmes reconnait avec un tolérence ajustable ( 0.1 par default) les amorces puis ajuste le reads en retirant les nucléotides en excès. Lorsqu’aucune séquences d’amorces n’est retrouvé, le read est supprimé .

\subsubsection{Filtrage des qualités}

Les données provenant de séquenceur haut débit peuvent contenir de nombreuses erreurs de séquençage comparé au méthode classique comme la méthode Sanger. Ceci est particulièrement vrai avec la stratégie MiSeq et le kit 300pb. Il est donc important de supprimer les reads de mauvaise qualité pour gagner en spécificité.
Tout d’abord une analyse évaluant la qualité des reads à été réalisé sur chaque échantillon avec le logiciel FastQC. Ce programme produit à partir des fichiers fastq une série de graphiques permettant de juger sur la qualité du séquençage.
Après cette analyse, le filtrage des reads de mauvaise qualité est exécuté en utilisant le programme sickle. L’alogirthme sous jacent repose sur une fenêtre glissante de taille défini ( par defaut : 20 pb) qui glisse tout le long de la séquence. A chaque étape la moyenne des scores de qualité est calculé dans cette fenêtre. Si successivement le score moyen dépasse un certain seuil, le reads est supprimé . Par default le seuil utilisé est de 20 avec une fenêtre glissante de x.
Enfin FastQC a de nouveau été lancé afin de comparer la qualité des reads avant et après filtrage.

\subsubsection{Assignement taxonomique}

L’assignement taxonomique consiste à labelliser chaque reads à son taxon. Pour cela deux stratégies existent.
La stratégie “de novo” consiste à regrouper les reads qui se ressemble en groupe ou cluster.
Chaque cluster définit alors une seul séquence consensus qui est comparé à une base de donnée d’ARN 16S pour recevoir son assignation taxonomique.
La stratégie “close reference” consiste à labélisé chaque reads en les comparant directement un par un à une base de donnée. Cette dernière stratégie peut sembler plus longue, mais elle est en réalité beaucoup plus rapide que la première stratégie. En effet la compléxité de la stratégie “de novo” est de type N ( N le nombre de reads) . Chaque reads étant comparé à l’ensemble des reads le temps de calcul augment de façon exponnentiel avec le nombre de reads.
La complexité de la deuxième stratégie est elle de type N. C’est à dire que le temps de calcul augmente avec le nombre de reads de façon linéaire. En contrepartie, si un read n’est pas retrouvé dans la base de donnée, celui ci est ignoré. Alors qu’en stratégie “de novo”, tous les reads herite du taxon de leurs clusters.
Les bactéries de la flores humaines étant plus étudié que d’autre flore plus exotique, elle sont très souvent retrouvé dans les bases de donnée. La stratégie close réference est donc suffisante dans le cas de notre étude avec des testes préliminiares montrant une assignation reussi chez plus de 98% des reads.

\section{Résultat}
\subsection{Pipeline Mucobiome}
Après demultiplexage, 188 x 2 fichiers fastq ont été généré soit 2 fichiers pairé par échantillons.
La taille des reads pour chaque fichier est de 301 paires de bases.
Au total 115’002’297 reads ont été produit sur 2 run MiSeq. Avec en moyenne 616900 reads par échantillon. Un minimum de 61422 reads pour l’echantillons 2154 et un maximum de 1071188 pour l’échantillon 3165.
Les analyses de qualité avec Fastqt montre dans l’ensemble une baisse de qualité en fin de séquence. Les 50 dernières bases, ont des scores de qualités médiocre entre 10 et 20 point selon le phred score.
Après prés traitement des reads, c’est à dire merging Flash et filtering 20, en moyenne 49.24 % des reads sont conservé avec des bornes allant de 37.30% à 61.13%.
L’assignement taxonomique a réussi sur 99.88% des reads.
Au total , le pipeline mucobiome s’est executé en 1h29 sur 40 coeurs et 20 giga de mémoire contre 42h dans les testes précédant sans optimisation.

\subsection{Diversité du microbiote réspiratoire}

\subsubsection{Courbe de rarefaction}

Les courbes de rarefaction par échantillons [figure] s’aplatisse précocement, témoignant d’un très bon niveau d’échantillonage.

\subsubsection{Diversité des échantillons}

Au total, 54 genres bactériens sont retrouvé dans l’ensemble des échantillons. [figure].
Les trois philums majoritaire, sont Protéobacteria(Haemophilus), Firmicutes(streptococcus) et Bacteroidete (prévotella). [figure pie].
Certaines genre bactérien sont très prévalente, c’est à dire présent dans l’ensemble des échantillons. Streptococcus, Neisseria, Prevotella, Granulicatella, Gemella, Veillonella et Fusobacterium sont présent dans plus de 185 échantillons.
D’autre sont dominante, c’est à dire qu’ils represente plus de 90% du microbiote pulmonaire dans certain échantillons. Il s’agit de streptococcus, Neisseria, Haemophilus et staphyloccoccus dans la majorité des cas. Sténotrophomonas et Achromobacter sont retrouvé dominante dans seulement 64 et 8 échantillons.
Le core microbiota est définit comme l’ensemble des taxons retrouvé dans plus de 50% des échantillons et ayant une abondance > 0.1%. Il est constitué de 15 genres décrit dans la figure [X].
La figure [violon] montre que Neisseria et Streptoccocus ont des abondances très variables dans les échantillons. Staphiloccocus et Haemophilus ont dans la plus part des échantillons des abondances relativement faible, mais la variance est tiré vers le haut par qque échantillons ou ils sont en dominance.

\subsection{Evolution dans le temps}
Alpha diversité (dot)
Abundance relatif (boxplot)

\subsection{Comparaison entre catégories Free et Never}

\subsection{Sensibilité et Specificté du Pyo}

\section{Discussion}


\section{Conclusion}





\newpage

\appendix

\newpage


\bibliographystyle{plain}
\bibliography{biblio}

\end{document}
