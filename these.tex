\documentclass[12pt,a4paper]{article}
    % for Nature's style citations
\usepackage[super]{natbib}
\usepackage[utf8]{inputenc}
\usepackage[greek,francais]{babel}
\usepackage[T1]{fontenc}
\usepackage{graphicx}
\usepackage{fancyhdr} % Needed to define custom headers/footers
\usepackage{setspace}
\usepackage{listings}
\usepackage{color}
\usepackage[scale=2]{ccicons}
\usepackage{caption}
\usepackage{url}
\usepackage[final]{pdfpages}
\usepackage{caption}
\usepackage{svg}
\usepackage{listings}
\usepackage{eurosym}
\usepackage{wrapfig}
\usepackage{subfig}
\usepackage[margin=2.7cm]{geometry}
    

    %%%%%%%%%%%%%%%%%%%%%%%%
    %%%%% MISE EN PAGE %%%%%
    %%%%%%%%%%%%%%%%%%%%%%%%
    %Interligne à 1.5
    \onehalfspacing





    \pagestyle{fancy} % Enables the custom headers/footers
    \lhead{Thèse de médecine }
    \chead{UBO}
    \rhead{Sacha SCHUTZ}
    \lfoot{truc}
    \rfoot{2015/2016}
     
     
    %%%%%%%%%%
    % MACROS %
    %%%%%%%%%%
    \newcommand{\HRule}{\rule{\linewidth}{0.5mm}} % Defines a new command for the horizontal lines, change thickness here
    \newcommand\nt{nucléotides }
     
    \newcommand{\includefigure}[3] {% label, caption, width ratio
    %ex\includefigure{RkNN}{Graphe d'exemple et R1NN/R2NN associés.}{0.8}
            \begin{center}
         \includegraphics[width=#3\textwidth]{#1}
         \captionof{figure}{#2}
         \label{FIG:#1}
            \end{center}
    }
     


\begin{document}


%%%%%%%%%%%%%%
% TITLE PAGE %
%%%%%%%%%%%%%%
\begin{titlepage}
\center







%       HEADING SECTIONS
\textsc{\LARGE Thèse de médecine}\\[1.5cm]
\textsc{\Large CHU-BREST}\\[0.5cm]
\textsc{\large Université de Brest\\
Brest \
}\\[0.5cm]
 
%       TITLE SECTION
\HRule \\[0.8cm]

{ \huge \bfseries Description du microbiote pulmonaire chez les patients atteints de mucovicidoses}\\[0.4cm]

\HRule \\[1.2cm]
 
%       AUTHOR SECTION
\begin{minipage}{0.4\textwidth}
 \begin{flushleft} \large
     \emph{Auteur:}\\
     Sacha SCHUTZ
 \end{flushleft}
\end{minipage}
~
\begin{minipage}{0.4\textwidth}
 \begin{flushright} \large
     \emph{Responsable:} \\
     Geneviève HERY-ARNAUD
 \end{flushright}
\end{minipage}\\[2cm]
 
{\large \today}\\[8cm] % Date, change the \today to a set date if you want to be precise
 
% LOGO SECTION
\begin{minipage}[c]{0.3\textwidth}
   \includegraphics[width=0.7\textwidth]{img/logo_brest.jpg}\hfill
\end{minipage}
\begin{minipage}[c]{0.3\textwidth}
   \includegraphics[width=0.7\textwidth]{img/ubo.png}
\end{minipage}
%\begin{minipage}[c]{0.3\textwidth}
%        \includegraphics[width=0.7\textwidth]{img/logo_brest.jpg}     
%\end{minipage}
% 
 
 
 
\vfill % Fill the rest of the page with whitespace

\end{titlepage}


 %----------------PAGE ENGAGEMENT ET LICENSE ---------------------------

\newpage

\section*{Engagement de non plagiat}

Je, soussigné Sacha SCHUTZ, interne en biologie moléculaire au CHU de Brest, déclare être pleinement informé que le plagiat de
documents ou de parties de documents publiés sur toute forme de
support, y compris l'internet, constitue une violation des droits
d'auteur ainsi qu'une fraude caractérisée.

En conséquence, je m'engage à citer toutes les sources que j'ai
utilisées pour la rédaction de ce document.

Date : 17/05/2017

\vspace{0.5cm}

Signature : \\

ssh pub key fingerprint : a4:e3:da:87:78:2d:e1:6f:bb:56:5c:d1:72:f5:50:63
\vfill 

\section*{License}

\begin{wrapfigure}{R}{0.3\textwidth}
\includegraphics[scale=0.5]{img/gfdl.png}\hfill
\end{wrapfigure}

Copyright (c) 2015 SCHUTZ Sacha. Permission est autorisée de copier,
distribuer et/ou modifier ce document sous les termes de la Licence de
Documentation Libre GNU, Version 1.2 ou toute version ultérieure publiée
par la Free Software Foundation ; with no Invariant Sections, no
Front-Cover Texts, and no Back-Cover Texts. Une copie de la license est
incluse dans la section intitulée ``GNU Free Documentation License''.

 %----------------PAGE REMERCIEMENT: A FAIRE.. ---------------------------
\thispagestyle{empty} 
\setcounter{page}{0}
\thispagestyle{empty} 

\newpage

\tableofcontents
\newpage


\section{Avant-propos}

Depuis Pasteur, les micro-organismes ont toujours été perçus négativement car associés aux maladies. La mise en évidence des agents pathogènes allant de la syphilis jusqu'au grande peste n'a pas aidé ces êtres microscopiques à sortir de ce stéréotype. Ceci a naturellement orienté la médecine à les combattre plus qu'à les étudier.
Aujourd'hui, personne ne peut nier que la médecine anti-infectieuse a permis l'amélioration de notre santé. 
Les avancées majeures en ce qui concerne l'hygiène, la vaccinations et les antibiotiques nous ont permis de diminuer la prévalence des maladies infectieuses  et pour certaine les faire disparaître. Cependant, la destruction systématique et massive des micro-organismes qui vivent et évoluent en nous depuis des milliers d'années pourrait bien être la cause de l'émergence de nouvelles maladies comme les maladies auto-immunes( Figure \ref{hyigienisme}).


\begin{figure}[ht]
\begin{center}
\includegraphics[scale=0.5]{img/allergie_infection.jpg}\hfill
\end{center}
\caption{Incidence des maladies infectieuses et autoimmune en europe au cours du temps. [@]}
\label{hyigienisme}
\end{figure}


Les nouvelles méthodes d'exploration de ce monde microscopique, comme le séquençage haut-débit ont permi aux bactéries de retrouver leurs lettres de noblesse.
Les bactéries sont retrouvées partout et s'accaparent le premier rôle dans le fonctionnement des écosystèmes. Elles sont par exemple impliquées dans le cycle de l'azote en permettant à la biomasse d'absorber l'azote athmosphérique. Les bactéries sont dans ce sens, la source primaire permettant aux organismes de construire leurs protéines et leurs ADN.
Elles peuvent vivre dans les milieuz les plus inhospitaliers. Les archés (anciennement archébactérie) peuvent résister à des conditions d'acidités et de températures exceptionnelles. On les retrouve dans les fonds océaniques où privées de lumière, elle sont la seul source d'énergie pour la faune en utilisant la chimiosynthèse à l'instar de la photosynthèse. les archés nous ont par ailleurs éclairés sur l’origine des eucaryotes \footnote{Le génome des arché est composé d'intron comme chez les eucaryotes} et nous ont permise un bon de géant en biologie moléculaire \footnote{La Taq polymérase est une enzyme d'arché résistant à de haute température utilisée dans les PCR}.\\
L'homme ne fait pas exception. La majorité des bactéries ont longtemps été in-détectable par les méthodes de culture classique. Mais à présent, les régions anatomiques autrefois considérées stériles foisonnent de bactéries. 
Elles sont retrouvées dans toutes les régions du corps exposées ou elles forment des communautés.
La peau est colonisée par \textit{Propionibacterium}, \textit{Corynebacterium} et \textit{Staphylococcus} \citep{Beck}. Le vagin contient des \textit{Lactobacille} et la bouche principalement du \textit{Streptococcus}\cite{Beck}.
L'intestin est une flore bactérienne dominée par les anaérobies pouvant représentées jusqu'à 2 kg du poids corporel \citep{Beck}.
En échange de son hospitalité, le microbiote participe au bon fonctionnement de son hôte. Il aide à la digestion en dégradant par exemple les sucres du lait maternelle chez le nouveau née[@bifidus]. Il participe à la synthèse de vitamine essentielle (K, B12,B8)[@ref]. Il éduque notre système immunitaire et fait barrière à tout nouvel agent pathogène.
Toute défaillance de notre microbiote ou \textit{dysbiose}, peut être délétère pour notre santé. La liste des maladies associées est longue. On retrouve par exemple la maladie de Crohn[@], la maladie coeliaque[@], le cancer de l’intestin[@], le syndrome du colon irritable[@], l’obésité[@], le diabète de type 1[@], l’asthme[@], l’eczéma[@], la sclérose en plaque[@], la polyarthrite rhumatoïde[@], la maladie d’alzheimer[@] et même l’autisme[@]. \\
La colite à \textit{Clostridium Difficile} est un exemple de dysbiose avec une application clinique directe. Suite à un traitement par antibiotique, la flore intestinale est détruite laissant alors l'opportunité à \textit{Clostridium Difficile} de s'installer. Un des traitements proposé est la transplantation fécale visant à régénérer le microbiote du patient. \\
Le microbiote amène donc à reconsidérer notre individualité. Nous ne sommes plus suelement qu'un eucaryote multicellulaire composé d' un seul génome. Mais plutôt un écosystème ou cellules eucaryotes et microbiennes vivent en symbiose. Cette relation n'étant pas figée dans le temps et pouvant varier entre le commensalisme, la parasitisme et le mutualisme. Les dernières études estiment que pour chaque individu il y a environ 30 billions de cellules humaines pour 39 billions de cellules microbiennes[@]. En associant les gènes bactériens, le génome d’un individu passe de 23 000 gènes à 3,3 millions[@] de gènes avec toutes la complexités des interactions que cela engendre[@]. Les scientifiques ont donné le nom d’\textit{holobionte} à cette écosystème vivant. L'ensemble du génome humain et microbien est appelé \textit{hologénome}. \\
Il faut toutefois rester prudent quant au rôle donné aux microbiotes et éviter de tomber dans l'excès. Nombreuse sont les publications scientifiques qui se contredisent ou qui confonde corrélation et causalité. L'excès de publication à même conduit à la création du hashtag humouristique sur twitter: \textit{\#GutMicrobiomeAndRandomSomething}. 
Les études sur le microbiote nécessitent d'être étayer par la métagénomique fonctionnelle afin de trouver les relations de cause à effet. Les corrélations doivent être réalisées sur des populations plus grande avec un suivi dans le temps plus important. Les nouvelles technologies de séquençage haut débit vont dans ce sens en collectant toujours plus de donnée.\\
Il est encore trop tôt pour dire si cette science va révolutionner la médecine de demain ou si il s'agit d'un effet de mode. Mais au regard de l'evolution biologique il y a fort à parier dessus. Car ne l'oublions pas, ce sont bien des anciennes bactéries, qui permettent à l'ensemble de nos cellules de respirer et que nous appelons maintenant des mitochondries.

\begin{figure}[ht]
\begin{center}
\includegraphics[scale=0.5]{img/mitochondrie.jpg}\hfill
\end{center}
\caption{La mitochondrie est l'exemple de symbiose ultime entre eucaryote et procaryote}
\label{bach}
\end{figure}



\newpage

\section{Définition}

\begin{description}
  \item[Le microbiote] est l’ensemble des micro-organismes (bactéries, levures, champignons, virus) vivant dans un environnement donné.
  
\item[Le microbiome] s’emploi selon deux définitions. En français, le microbiome est l'environnement qui héberge le microbiote. Dans sa définition angle-saxonne, le microbiome fait référence à l’ensemble des génomes microbiens contenus dans un environnement. 
De façon général le microbiome est associé aux génomes bactériens. Les termes de Virome et de Mycobiome sont utilisés pour les génomes viraux et myoctiques. 

\item[La biocénose] est le terme écologique dans un sens large désignant l'ensemble des organismes vivants dans un environnement appelé \textbf{Biotope}. Biocénose et biotope forme ensemble un \textbf{écosystème}.

\item[Une symbiose] est une association durable entre deux organismes. Leurs relations peuvent être mutualiste, parasitaire ou commensale.

\item[La métagénomique] est une méthode d’étude du contenu en ADN présent dans un milieu grâce aux techniques de séquençage haut débit. Contrairement à la génomique qui s’intéresse au génome d’un individu, la métagénomique s’intéresse aux génomes d’une population d’individu.
Dans son sens stricte, la métagénomique correspond à l’étude de l’ensemble des séquences d'ADN. L’analyse d’un seul gène, comme celui de l’ARN 16s est associé à tord au terme métagénomique, mais son usage reste courant. On lui préférera le terme de \textbf{metagénétique}

\item[Un read] est un terme bioinformatique désignant une séquence d’ADN issue d’un séquençage haut débit. Selon les technologies, les reads varient entre 150 et 300 paires de bases.

\item[Un OTU] (\textit{Operational taxonomic Init}), est un terme utilisé en phylogénie, désignant un groupe d’individu proche faisant souvent référence à l’espèce dans la classification de Linée.
En microbiologie, un OTU est défini par un groupe d’individu ayant une similarité dans leurs séquence d'ARN 16s supérieur à 97\%.

\item[L'Abondance] absolue est le nombre de séquences d'ADN d'un OTU retrouvé dans un échantillon. 
L’abondance relative est le pourcentage en séquences d'ADN d'un OTU retrouvé dans un échantillon. Ce dernier permet de rendre les échantillons comparables entre eux.

\item[La table des OTU] correspond à un tableau à double entrée contenant l’abondance par OTU  et par échantillon. Dans le tableau suivant, l'échantillon 1 contient 68\% de l'OTU 1.

\begin{figure}
\begin{center}
\begin{tabular}{|l|c|c|c|c}
  \hline
   & échantillon 1 & échantillon 2 & échantillon 3  \\
  \hline
  OTU 1 & 68\% & 12\% & 25\% \\
  OTU 2 & 40\% & 24\% & 25\% \\
  OTU 3 & 28\% & 64\% & 50\% \\

  \hline
\end{tabular}
\end{center}
\caption{La table des OTUs}
\end{figure}

\item[La diversité alpha] est une mesure de biodiversité au sein d’un échantillon. Elle correspond à l’étude d’une colonne dans la table des OTUs. Plusieurs indicateurs de diversité alpha existent.

\item[La diversité beta] est une analyse descriptive de la biodiversité entre plusieurs échantillons. Elle correspond à l’étude de l’ensemble de la table des OTUs. L’approche la plus courante est de réaliser une analyse multivarié par des méthodes d’ordination. Il s’agit de représenter un graphique à N dimensions, impossible à dessiner, en le projetant dans un espace à une ou deux dimensions.

\item[La richesse] est le nombre d’espèce présent dans un échantillon. Les deux échantillons suivant une richesse de 2.

échantillon 1  : 4 Streptoccus , 4 Escherichia  \\ 
échantillon 2 : 432 Streptoccus, 12 Escherichia 

\item[L'uniformité / equitabilité] indique si les espèces d’un échantillon sont réparties uniformément.
L'uniformité du premier échantillon est plus grande que la seconde

échantillon 1  : 50 Streptococus , 50 Escherichia  \\ 
échantillon 2 : 432 Streptoccus, 12 Escherichia 


\item[L'indice Chao1] est une estimation de la richesse réelle (in vivo) par rapport à la richesse observée (in vitro). Cette indice part du principe que si l’échantillon contient beaucoup de singletons ( OTU détecté une seul fois), il est fort probable que la richesse réel soit plus grande que la richesse de l’échantillon. La formule est la suivante.
\begin{equation}
A = B
\end{equation}

\item[L'indice de Shannon] est un indicateur évaluant à la fois la richesse et l’uniformité dans un échantillon. Il se calcul de la même façon que l’entropie de Shannon.

\begin{equation}
A = B
\end{equation}

\item[L'indice de Simpson] est un indicateur évaluant la probabilité que deux individus sélectionnés aléatoirement dans un échantillons donnée soient de la même espèces. La formule est la suivante.

\begin{equation}
A = B
\end{equation}

\item[Pipeline] 
Un pipeline est un ensemble d'étape de calcul. Chaque étape prend en entrée des fichiers pour en produire des nouveaux dans sa sortie. On peut comparer cela aux étapes d'une recette de cuisine. Sans parallélisation, un cuisinier (le processeur) doit attendre de faire fondre le beurre avant de battre les oeufs en neige ( Exécution synchrone). En parallélisant, le cuisinier peut réaliser plusieurs étapes en même temps. Battre les oeufs pendant que le beurre fondent.( Exécution asynchrone). 
Maintenant si l'objectif est de produire 188 gâteaux (188 analyses) et que l'on dispose de 64 cuisiniers ( 64 processeurs), l'organisation des tâches devient complexe si l'on veut maximiser le rendement. Pour cela, on dispose d'outil comme snakemake, qui permette de générer un graphe des étapes( Direct Acyclique Graph) et de trouver la meilleur façon d'optimiser les taches entre les differents cuisiniers (processeur). 


\item[La courbe de rarefaction] est utilisé pour déterminer si la profondeur de séquençage est suffisante pour caractérisé la diversité d’un échantillon.
Pour générer cette courbe, des groupes de reads de taille croissante (1…n) sont tiré aléatoirement sans remise. Le groupe est reporté sur l'axe X et le nombre d’OTU correspondant est reporté sur l’axe Y.
Une courbe s’aplatissant indique une bonne profondeur de séquençage \citep{Dickson2014} and \citep{Beck}

\end{description}

\newpage

\setcounter{page}{1}

\section{Introduction}
\subsection{La mucovicidose}
\subsubsection{Une maladie génétique}
La mucovicidose est une maladie génétique autosomique récessive grave touchant en France 1 naissance sur 5400 [@]. La Bretagne est la région la plus touchée avec une prévalence de 1/3000[@].
La loi de Hardy Weinberg estime qu’en Bretagne 1 patient sur 25 est porteur de la mutation à l’état hétérozygotes[@Heterozygote advantage]. Cette haute prévalence s’explique probablement par un effet fondateur associé à un avantage sélectif pour les individus porteur de l’allèle muté. \footnote{Plusieurs hypothèses ont été proposé, notamment lors des grandes épidémies de choléra en diminuant les pertes hydriques. D’autres suggère qu'il s'agit d'une pleiotropie antagoniste.[@]} \\
Le gène CFTR impacté se situe sur le chromosome 7 en position q31.2. Il est constitué de 27 exons pour 250,188 [@] paire de bases. Il code pour une canaux chlore AMp dépendant permettant les échanges des ions chlorures au niveau des membranes cellulaires[@]. Il est également impliqué dans le transport du thiocynate (SCN-) et des bicarbonate (HCO3-)[@]. \\
On dénombre à ce jour 2017 mutations impliquées dans la mucovicidoses[@]. La perte d’une phénylanine en position 508 par délétion du triplet c.1521-1523delCTT (anciennement $\Delta$F508) est responsable à elle seul de 80\% des mucovicidoses.
Ces mutations sont responsable d’une protéine défectueuse ou d’une absence de canaux sur les membranes cellulaires. \\
Cliniquement, la mutation est responsable chez les patients d’une insuffisance pancréatique exocrine et d’une infertilité par disparition des canaux déférants. Des signes digestifs, hépatiques et articulaires sont également retrouvés.
L'atteinte de la fonction respiratoire est la plus bruyante. En effet au niveau de l’épithélium broncho-pulmonaire, l’absence d’un CFTR fonctionnel est à l’origine d’une déshydration du mucus le rendant plus visqueux et empêche les cils bronchiques de jouer leurs rôles.[@]\\
La forte prévalence de la maladie nécessite de réaliser un dépistage précoce chez tous les nouveaux nées (test de Gutri) afin d’adapter au plus tôt la prise en charge. Seul le test à la sueur permet de poser le diagnostic. Des tests de dépistage prénatal basés sur l’ADN circulant sont actuellement à l’étude[@]. Le traitement repose avant tout sur une prise en charge respiratoire (kinésithérapie, dornase, antibiothérapie).
Les thérapies génétiques sont encore à l’étude[has]
L’Ivacaftor est le seul traitement à ce jour qui agit directement sur le CFTR. Mais concerne uniquement certaine mutation rare, comme la G551D.[has]La greffe pulmonaire est le dernière recours.

\subsubsection{Une maladie infectieuse}

L’atteinte pulmonaire est caractérisé par des infections successives associées à une réaction inflammatoire qui dégrade progressivement la fonction respiratoire.
Plusieurs pathogènes sont impliqués. Chez les jeunes enfants, \textit{Haemophilus influenza} et \textit{Staphilococcus Aureus} sont majoritairement retrouvés. \textit{Burkolderia Cepace} et \textit{Stenotrophomonas Maltophilia} sont retrouvés chez le sujet plus âgé.
Mais c’est \textit{Pseudomonas Aeruginosa} qui caractérise l’atteinte pulmonaire dans la mucovicidose en marquant un tournant décisive dans l’évolution de la maladie. Ce bacille aérobie stricte, est un germe de l'environnement rarement retrouvé chez les patients sains[@]. En revanche il est retrouvé chez 60\%[@] des patients jeunes, et plus de 90\% des patients adultes[@].\\
La primocolonisation à \textit{Pseudomnas Aeruginosa} est difficilement détectable, mais semble avoir lieu tôt dans l’enfance[@]. Il y a ensuite une phase de latence, variable entre les individus, marquée par des épisodes d’exacerbations e de rémissions. A ce moment l’éradication[@] par des antibiotiques reste possible.
Puis survient le passage à la chronicité. \textit{Pseudomnas Aeruginosa} s'adapte à son milieu et s’installe à long terme. Il perd certain caractère de virulence, mais devient résistant aux antibiotiques[@]. Son phénotype change. Il devient mucoïde en sécrétant un film d’alginate qui le protège du système immunitaire. Les mécanismes sous jacent à cette transformation sont ingénieux. La forte densité en bactérie est responsable d’activation de certain gène par un processus appelé \textit{quorum sensing}[@]. un processus dans lequel chaque bactérie communique avec ses voisines via des signaux.
Le génome de \textit{Pseudomnas Aeruginosa} devient aussi hypermutable afin de présenter une plus grande diversité génétique au regard de la sélection naturelle\footnote{En biologie évolutif, il s'agit d'évolvabilité}. \\
A ce stade le traitement antibiotique n’est plus curatif et l'évolution tend inexorablement vers un déclin de la fonction respiratoire.
L'approche clinique est donc préventif. Elle vise à éliminer \textit{Pseudomnas Aeruginosa} dès qu'il est détecté en culture. Une surveillance rapproché des patients avec un prélèvement mensuel ou bimensuel est préconisé selon l'HAS[@]. La culture étant peu sensible, d’autres méthodes de détections peuvent être employées. La détection des anticorps anti-pyocianique par des méthodes elisa a montrer peu de sensibilité [@]
La PCR ciblé associant les protéines bactériennes \textit{OPRL1} , \textit{GYRB1} et \textit{ECFX1}  s’est montré plus sensible et plus spécifique que la culture[@]. \\
En pratique, la colonisation chronique est définie lorsque 3 expectorations sont rendues positives en culture, successivement au cours d’un suivi mensuel ou bimensuel[@].
Une autre classification, celle de Lee a montrer une forte liaison clinico-biologique. Elle est composé de 4 groupes :  
\begin{description}
\item[groupe chronique] > 50\% des cultures sont positives sur 12 mois
\item[groupe intermediaire] $\leq$ 50\% des cultures sont positives sur  12 mois
\item[groupe Free] Toutes culture négative sur 12 mois, avec des antécédants
\item[groupe Never] Toutes culture négative sur 12 mois sans antécédants 
\end{description}

On ne sait pas aujourd’hui pourquoi \textit{Pseudomnas Aeruginosa} s’installe préférentiellement chez les patients atteint de mucoviscidose. Plusieurs hypothèse ont été proposé : 

%----------------- REFORMULER -----------------
\begin{itemize}
\item La dysfonction cilliaire empêche les Pseudo d’etre viré par le haut
\item L’hypersalinité du film muqueus desactive les peptides antimicrobien
\item Le CFTR est un récepteur de Pyo qui les internalise et les viré
\item L’inflammation de  l’epithelimum augmente les metabolite qui permette de se developper.
\item Alanine et lactta sont une source de carbonne pour le Pyo.
\item Le microbiote influence la colonisation
\item Le pyo stimule le Systeme immunitaire pour virer tous les autres concurrant 
\end{itemize}




\begin{figure}[ht]
\begin{center}
\includegraphics[scale=0.8]{img/chronic.png}\hfill
\end{center}
\caption{Infection pulmonaire à pyo dans la mucovicidose. ref []}
\label{bach}
\end{figure}


\subsection{Le microbiote pulmonaire}

Bien qu'il soit en contacte avec le milieu extérieur, l’arbre respiratoire (comprenant la tranché, les bronches et les alvéoles) a longtemps été considéré comme stérile avec les méthodes de culture classique. Il a fallu attendre l’avènement du séquençage haut débit pour mettre en évidence le microbiote pulmonaire[ref Host-microorganism 1-3].
Le microbiote pulmonaire est beaucoup moins abondante que la flore digestive. Il est constitué d’une flore dynamique provenant de l’air ambiant, des voies supérieurs  mais aussi du tube digestif via des micro-aspirations.[@]
Le microbiote pulmonaire est dominée par le phylum des Firmicutes ( Streotpococcus) et des Bacteroidetes (Prevotella). Les genres retrouvés majoritairement sont Streptococcus, Prevotella, Fusobacteria, Veillonella, Haemophilus, Neisseria et Porphyromonas.
l’arbre respiratoire étant en continuité direct avec les voies aérienne supérieur, certain genre bactérien sont commun, comme Streptococcus, staph, Haemophilus et Moraxella. Tandis que d’autre genre comme corynebacterium et Dolosigranulum ne sont retrouvé qu’au niveau du nez et de l'oropharynx.

Le microbiote pulmonaire varie dans l’espace et le temps. \\
Dans l’espace, du fait de sa structure, certaines régions de l'arbre bronchique peuvent présenter des microbiotes différents. Un prélèvement d'un foyer infectieux sera nécessairement différent d'un foyer sain. Le poumon présente également des différences physico-chimiques selon la localisation pouvant sélectionner certaine espèce. Les cavernes tuberculeuses par exemple se trouve essentiellement dans le lobe supérieur en raison d'une concentration en oxygène plus élevé favorisant ce bacille aérobie stricte. 
Le microbiote varie dans le temps.......résilience .............
Le microbiote est variable entre les individus ... 
Le microbiote est corrélé à la pathologie respiratoire. Plusieurs études suggère une différence entre patients sains et patient asthmatique, BPCO ou atteint de mucovicidose [@].

\subsection{Exploration du microbiote pulmonaire}
%(http://bacterioweb.univ-fcomte.fr/bibliotheque/remic/08-Bronc.pdf) ==> A LIRE COOURS BACTERIO

\subsubsection{Méthode de prélèvement}
Le microbiote respiratoire est explorer en sequençant l'ensemble des ADNs présent dans un échantillon. 
Toutes les méthodes de recueils sont possible, mais les prélèvement protégés (combicath, LBA) sont recommandés afin d’éviter une contamination par les voies supérieurs. Dans le cas contraire (ECBC) on peut évaluer la qualité du prélèvement en comptant le nombre de cellule épithélium ( normalement bas ) et de polynucléaire ( normalement haut ) dans le poumon. La meilleurs méthode de prélevement étant le prélèvement in situ réalisable lors des greffes pulmonaires.

\subsubsection{Séquençage haut débit}
Le séquençage haut débit permet de sequencer l'ensemble des ADNs présent dans un microbiote. Il est ainsi possible de déterminer sa composition et sa fonction. A titre d'exemple, un séquenceur Sanger classique permet de lire des fragments d'ADN d'environ 800 pb. Cette étape est parallélisable jusqu'à 96 fois en augmentant le nombre de capillaires.
A l'inverse un séquenceur de nouvelle génération lit des fragments plus court de l'ordre de 150pb. Mais la parallélisation est remarquable allant jusqu'20 billion de fois avec un seul run sur l'illumina Novaseq. \\
2 strategies de séquençages sont possible :  \\
Le stratégie shotgun consiste à séquencer l'ensemble des ADNs présents dans l'échantillon sans discernement, que ce soit humain ou bactérien. Les séquences sont filtrer puis les génomes bactériens sont reconstruit par des méthodes bioinformatiques complexes. \\
La deuxième stratégie est moins coûteuse en terme d'analyse. Elle consiste à amplifier un gène n'appartenant qu'aux bactéries et suffisamment variable pour discriminer une espèces. Pour les bactéries il s'agit de l'ARN 16S.\\

\subsection{ARN 16S}

L'ARN 16S est un ARN non codant participant à la structure de la petite sous unité des ribosomes bactériens. Il est composé de 1500 nucléotides et forment plusieurs boucles dans sa structure secondaire (Figure \ref{ARN16S}). 
L'alignement des séquences d'ARN 16S entre plusieurs espèces met en évidence des régions constantes et variables Figure \ref{ARN16SVariation}). Les 9 régions variables appelées V1 à V9 peuvent être amplifiées puis séquencées afin d'identifier l'espèce correspondante.

\begin{figure}[ht]
\begin{center}
\includegraphics[scale=0.8]{img/ARN16S_variation.png}\hfill
\end{center}
\caption{région constante et variable de l'ARN 16S}
\label{ARN16SVariation}
\end{figure}

\begin{figure}[ht]
\begin{center}
\includegraphics[scale=0.8]{img/ARN_16S.png}\hfill
\end{center}
\caption{Structure secondaire de l'ARN 16S}
\label{ARN16S}
\end{figure}


La première stratégie est plus informative car elle séquence l'ensemble des génomes bactériens et permet de prédire la fonction d'un microbiote. En effet les transferts génétiques horizontaux amènent à dissocier l'espèce de sa fonction. 2 bactéries d'une même espèce peuvent avoir des fonctions différentes. L'inférence fonctionnelle réalisé à partir de la stratégie 16S est déconseillé[@].  
La stratégie 16S reste toutefois une méthode simple pour décrire les populations bactériennes présentent. C'est cette stratégie qui a été utilisé dans notre étude. 



\subsection{Objectif de l'étude Mucobiome}
L'objectif de notre étude est de savoir si le microbiote respiratoire influence la primo-colonisation à \textit{Pseudomnas Aeruginosa}. 
Pour cela, nous avons suivit pendant 3 ans,  une cohorte de 47 patients atteints de mucovicidoses sans infection chronique. 
L'exploration de leurs microbiotes par la stratégie 16S a été réalisé sur leurs ECBC dans le cadre de leurs suivis réguliers. 
A partir des données générés du séquençage, nous avons réalisé l'étude descriptive et analytique de leurs microbiotes en créant un pipeline bioinformatique dédié. 

\section{Matériel et Méthodes}
\subsection{Recueil des données}

47 patients atteints de mucovicidose ont été suivit sur 3 ans (2008-2011) dans une étude prospective multicentrique (Nantes,Brest,Roskoff) appelé mucobiome.
La CPP VI-Ouest et le comité d’éthique du CHRU de Brest ont approuvé le protocole. Tous les patients ( ou les parents pour les mineurs) ont signé un consentement éclairé. Le protocole à fait l’objet d’une déclaration de biocollection à l’ARS et au MESR (n DC-2008-214).\\
les ECBC des patients ont été recueillies lors des séances de kinésithérapie respiratoire tous les 3 mois, suivant le calendrier des recommandations officielles. En pratique, sur l’ensemble de la cohorte suivie, l’intervalle médian entre 2 consultations a été de 3.4 mois.
Les patients devaient avoir un génotype CFTR et un test à la sueur positif. Les transplanté ont été exclus de l’études.
Une culture positive à \textit{Pseudomnas Aeruginosa} était un critère de non-inclusion. Si pendant l’étude, une culture revenait positif à ce dernier, le patient était sortie de l’étude pour être réinclus 1 ans après en l’absence de colonisation chronique. 15 patients ont été ainsi ré-inclus.
Chaque patient a été classé dans la catégorie Free ou Never (Lee et all 2003). D’autres données ont été également recueilli ( tableau).
Au total , 188 échantillons ont été récolté, soit en moyenne 4 échantillons par patients.
Pour chaque échantillon, une culture a été réalisé en suivant les procédures standards [ref]. Une qPCR ciblant le \textit{Pseudomnas Aeruginosa} a également été réalisé en combinant les marqueurs gyrB/ecfX designé au laboratoire[ref].


\begin{figure}[ht]
\begin{center}
\includegraphics[scale=0.8]{img/summary.png}\hfill
\end{center}
\caption{Données associés par échantillons ( a refaire par patients)}
\label{summary}
\end{figure}



\subsection{Extraction de l’ADN}

Les échantillons ont été liquéfiés avec du Dithiotrétiol. Les protéines ont été degradés avec une Protéine kinase.
Les parois bactériennes ont été fragmenté par sonication. (DTTpar sonication (Elamsonic S10, Singen, Germany). Après 10 min de centrifugation, L’ADN a été extrait à partir du culot via QUIAamp DNA Minikit ( Quagen).
Les extraits d’ADN ont été envoyé pour séquençage via un prestataire GATC.

\subsection{Séquençage}
La librairie \footnote{l'ensemble des fragments d'ADN à séquencer} a été produit en amplifiant la région V3-V5 à l’aide du couple d’amorces  \textit{forward(CCTACGGGAGGCAGCAG)} et \textit{reverse(CCGTCAATTCMTTTRAGT)} et du kit MiSeq Reagent Kits v3. \\
La séquençage a été produit sur Illumina MiSeq. Cette technologie permet de lire un couple de séquence de  300pb correspondant à l'amplicon lu par ses deux extrémités. La figure \ref{illumina} montre le chevauchement du couple de reads permettant de lire l'amplicon V3-V5 de 535pb. \\
Environ 25 millions de reads sont produit par run. En multiplexant à l’aide de 94 index, 2 runs ont permis de séquencer les 188 échantillons.
Au final 188 x 2 fichiers fastq ont été générés à l’issue du séquençage.

\begin{figure}[ht]
\begin{center}
\includegraphics[scale=0.6]{img/illumina.png}\hfill
\end{center}
\caption{le couple de séquence de 300pb permet de recouvrir l'ensemble de la région V3-V5}
\label{illumina}
\end{figure}


\subsection{Analyse bioinformatique}

l’analyse des 188 x 2 fichiers fastq a été réalisé grâce aà un pipeline bioinformatique, appelé \textit{mucobiome}, conçu et tester dans le cadre de cette étude. Par rapport aux autres logiciels comme \textbf{QIIME} ou \textbf{MOTHUR}, le pipeline mucobiome est spécialisé dans l’analyse des données 16S. Il est également plus rapide en raison d’un très haut niveau de parallélisation permis grâce à  \textbf{Snakemake}. Cette outil modélise l'ensemble du pipeline sous forme d'un graphe direct acyclique (DAG) et le résoud afin d'optimiser le parallélisation. \\
Le pipeline mucobiome prend en entrée, les 188 fichiers fastq provenant du séquençage et produit un fichier BIOM contenant la table des OTUs. La figure \ref{pipeline_trio} et \ref{dag} résume l'ensemble du pipeline.

\begin{figure}[!ht]
\begin{center}
\includegraphics[scale=0.5]{img/pipeline_trio.png}\hfill
\end{center}
\caption{Graphe du pipeline simplifié sur 3 échantillons 1015,1001 et 1003.\\ \textbf{merging}: Les reads pairés de 300 pb sont fusionnés  pour produire un fichier fastq contenant des reads de 535pb. \textbf{cleanning}: Les reads de mauvaise qualité sont supprimés. \textbf{reversing}: les reads sont transformé en leurs séquences complémentaires pour pouvoir être aligner. \textbf{trimming}: Seul la séquence entre les primers V3-V5 est conservés. \textbf{dereplicating}: Les séquences dupliquées. \textbf{merging}: L'ensemble des séquences est regroupé dans un seul fichier. \textbf{taxonomy assignment}: Les séquences sont aligné sur la base de données greengene. \textbf{create\_biom}: La table des OTU est crée }
\label{pipeline_trio}
\end{figure}


\begin{figure}[ht]
\begin{center}
\includegraphics[scale=0.4]{img/dag.jpg}\hfill
\end{center}
\caption{le graph du pipeline sur l'ensemble des échantillons}
\label{dag}
\end{figure}

% +++ RELECTURE 
\subsubsection{Pré-Traitement}
\subsubsection{Merging: Fusions des reads}\begin{center}\emph{ 2 fastq en entrée 1 fastq en sortie. } \end{center}

Les données bruts provenant du séquenceur sont des fichiers \textbf{Fastq}. Ils contiennent les séquences nucléotidiques et les scores de qualités par nucléotide lus par le séquenceur. Deux fichiers paired-end avec des reads de 300pb sont produit pour chaque échantillon. L’un correspond à la séquence lu dans le sens forward, l’autre dans le sens reverse.
La première étape du pipeline consiste à fusionner ces deux fichiers afin de produire une plus longue séquence de 500pb correspondant à la région V3-V5 de l’ARN 16S.
Le programme Flash[@] a été utiliser avec les paramètres par défaut. A partir de deux fichiers fastq, Ce dernier recherche le meilleur alignement entre deux reads et produit 1 fichier fastq contenant les reads fusionnées.
Une analyse qualitative des reads à été réalisé avec FastQt\citep{Beck} avant et après fusions.

\subsubsection{Cleaning: Filtrage des qualités}\begin{center}\emph{1 fastq en entrée 1 fastq en sortie. } \end{center}

Les données de séquençage haut débit peuvent contenir beaucoup d'erreur. Il est important de supprimer les reads de mauvaise qualité pour gagner en spécificité. ( expliquer la qualité)
Le filtrage des reads de mauvaise qualité est réalisé avec le programme sickle. Son algorithme repose sur l'utilisation d'une fenêtre glissante de taille défini ( par defaut : 20 pb). Cette fenêtre glisse le long de la séquence et pour chaque arrêt calcul la moyenne des scores de qualité. Si successivement le score moyen passe sous un certain seuil, le read est supprimé. Les paramètres utilisé sont ceux par défaut. Un score de 20 avec une fenêtre glissante de 20pb. 
Une analyse qualitative des reads à été réalisé avec FastQt\citep{Beck} après le filtrage. 


\subsubsection{Reversing: Séquence complémentaire}\begin{center}\emph{1 fastq en entrée 1 fasta en sortie. } \end{center}

Les reads produit par le séquenceur ne sont pas orienté dans le même sens que la base de donnée greengene. Pour permettre l'alignement, Les séquences ont été remplacé par leurs séquences complémentaire grâce au programme seqtk. 
Par la même occasion les scores de qualités devenu inutile sont supprimer. Les séquences sauvegarder dans un fichier fasta. 

\subsubsection{Trimming: Suppression des primers} \begin{center}\emph{1 fasta en entrée 1 fasta en sortie. } \end{center}

Pour permettre un alignement parfait entre les reads et la base de données les primers sont retiré et seul la séquence V3-V5 est conservée. Cette étape est réalisé aussi bien pour les données du séquençage que pour pour la base de données greengene.  
Le programme cutadapts est utilisé avec un tolérance de 0.1 par default. 

\subsubsection{dereplicating: Suppression des doublons}\begin{center}\emph{1 fasta en entrée 1 fasta en sortie. } \end{center}

Cette étape consiste à supprimer tous les reads dupliqués. En procédant ainsi, on s'assure de ne pas répéter l'assignement taxonomique plusieurs fois sur un même reads. C'est une étape d'optimisation permettant d'économiser en temps de calcul. La déréplication à été réalisé avec \textbf{vsearch} et sa fonction \textit{--derep\_fulllength }

\begin{figure}[ht]
\begin{center}
\includegraphics[scale=0.4]{img/dereplication.png}\hfill
\end{center}
\caption{Exemple de déréplication d'un fichier fasta}
\label{dag}
\end{figure}

\subsubsection{Assignement taxonomique} \begin{center}\emph{1 fasta et Greengene en entrée,  1 fichier biom  en sortie. }\end{center} 

L’assignement taxonomique consiste à labellisé chaque read à son taxon. Nous avons utilisé la stratégie \textit{close reference} qui consiste à comparer chaque read à une base de donnée avec un seuil de 98\% de similarité. Cette algorithme est de complexité N. C'est à dire que le temps de calcul est directement proportionnel au nombre de reads testé. La base de donnée Greengene version Mai 2013  a été utilisé. Elle contient 1'262'986 séquences et 203'452 OTUs. \\
L'autre stratégie d'assignement \textit{de novo} n'a pas été utilisé. Cette dernière, de complexité $N^{2}$, consiste à comparer les reads entre eux pour former des groupes. Elle s'emploie de préférence pour détecter les bactéries absentes des bases de données.
L'assignement taxonomique a été réalisé avec vsearch et sa fonction \textit{--usearch\_global}. 

\subsubsection{Analyse descriptive}
L'analyse de la table des OTUs a été réalisé avec R et le package phyloseq. 


\section{Résultat}

\subsection{Pipeline Mucobiome}
Après demultiplexage, 188 x 2 fichiers fastq ont été généré soit 2 fichiers pairé par échantillons.
La taille des reads pour chaque fichier est de 301 paires de bases.
Au total 115’002’297 reads ont été produit sur 2 run MiSeq. Avec en moyenne 616900 reads par échantillon. Un minimum de 61422 reads pour l’echantillons 2154 et un maximum de 1071188 pour l’échantillon 3165. La figure \ref{all} illustre ces grandes variations entre échantillons. \\
Les analyses de qualité avec Fastqt montre dans l’ensemble une baisse de qualité en fin de séquence. Les 50 dernières bases, ont des scores de qualités médiocre entre 10 et 20 point selon le phred score.\\
 La figure \ref{fastqcquality} represente le profil de qualité typique retrouvé. la figure \ref{fastqc merging} montre le profil de qualité des reads mergés avant et après filtrage des qualités.
Après traitement des reads, c’est à dire merging et filtering, en moyenne 49.24 \% des reads sont conservé avec des bornes allant de 37.30\% à 61.13\%. \\
99.88\% de l'ensemble des reads a reçu une assignation taxonomique malgrès la qualité. \\
Au total le pipeline mucobiome s’est executé en 1h29 sur 40 coeurs et 20 giga de mémoire contre 42h dans les testes précédant sans optimisation.

\begin{figure}[ht]
\begin{center}
\includegraphics[scale=0.45]{img/1003_forward.png}\hfill
\end{center}
\caption{Qualité par nucléotide des reads forward de l'échantillon 1003. \textbf{Axe X}: la position sur le reads. \textbf{Axe Y}: La distribution des qualités}
\label{fastqcquality}
\end{figure}


\begin{figure}[!ht]
\begin{center}
\includegraphics[scale=0.45]{img/duo_merging.png}\hfill
\end{center}
\caption{Qualité par nucléotide sur l'ensemble des reads forward de l'échantillon 1003}
\label{fastqc merging}
\end{figure}

\begin{figure}[!ht]
\begin{center}
\includegraphics[scale=0.25]{img/pipeline.png}\hfill
\end{center}
\caption{Nombre de reads analysables avant et après filtrage}
\label{all}
\end{figure}

\newpage

\subsection{Diversité du microbiote réspiratoire}
\subsubsection{Courbe de rarefaction}

Les courbes de rarefaction par échantillons (Figure \ref{rarefaction}) s’aplatisse précocement, témoignant d’un très bon niveau échantillonnage.

\begin{figure}[!ht]
\begin{center}
\includegraphics[scale=0.5]{img/rarefaction.png}\hfill
\end{center}
\caption{Courbes de raréfactions des 188 échantillons}
\label{rarefaction}
\end{figure}




\subsubsection{Diversité des échantillons}

54 genres (Figure \ref{genus})  et 7 phylums (Figure \ref{phylum}) bactériens sont retrouvé dans l’ensemble des échantillons. 
Les trois philums majoritaire, sont \textit{Protéobacteria}(\textit{Haemophilus}), \textit{Firmicutes}(\textit{Streptococcus}) et \textit{Bacteroidete} (\textit{Prevotella}). \\
Le tableau \ref{bigtable} montre la prévalence des genres bactériens dans les échantillons. 
Certaines genre bactérien sont très prévalente, c’est à dire présent dans l’ensemble des échantillons. \textit{Streptococcus}, \textit{Neisseria}, \textit{Prevotella}, \textit{Granulicatella}, \textit{Gemella}, \textit{Veillonella} et \textit{Fusobacterium} sont présent dans plus de 185 échantillons.
D’autre sont dominante, c’est à dire qu’ils represente plus de 90\% du microbiote pulmonaire dans certain échantillons. Il s’agit de \textit{Streptococcus}, \textit{Neisseria}, \textit{Haemophilus} et \textit{Staphyloccoccus} dans la majorité des cas. \textit{Sténotrophomonas} et \textit{Achromobacter} sont retrouvé dominante dans 64 et 8 échantillons. \\
Le core microbiota est définit comme l’ensemble des taxons retrouvé dans plus de 50\% des échantillons et ayant une abondance > 0.1\%. Il est constitué de 15 genres (Figure \ref{core}). \\
La figure \ref{violon} montre la variabilité des abondances entre échantillons. \textit{Neisseria} et \textit{Streptoccocus} ont des abondances très variables. \textit{Staphiloccocus} et \textit{Haemophilus} ont dans la plus part des échantillons des abondances relativement faible. Leur variance est tiré vers le haut par quelque échantillon ou ils sont dominant.


\begin{figure}[!h]
\begin{center}
\includegraphics[scale=0.5]{img/phylum.png}\hfill
\end{center}
\caption{truc}
\label{phylum}
\end{figure}

\begin{figure}[!h]
\begin{center}
\includegraphics[scale=0.5]{img/genus.png}\hfill
\end{center}
\caption{truc}
\label{genus}
\end{figure}

\begin{figure}[!h]
\begin{center}
\includegraphics[scale=0.5]{img/core.png}\hfill
\end{center}
\caption{truc}
\label{core}
\end{figure}




\begin{figure}[!h]
\begin{center}
\includegraphics[scale=0.5]{img/variability.png}\hfill
\end{center}
\caption{truc}
\label{violon}
\end{figure}



\begin{figure}[h]
\begin{center}
\includegraphics[scale=0.5]{img/all_table.png}\hfill
\end{center}
\caption{truc}
\label{bigtable}
\end{figure}

\subsection{Evolution dans le temps}


\begin{figure}[h]
\begin{center}
\includegraphics[scale=0.5]{img/all_alpha_shanon.png}\hfill
\end{center}
\caption{truc}
\label{shannon}
\end{figure}


\begin{figure}[h]
\begin{center}
\includegraphics[scale=0.5]{img/evolution_Core.png}\hfill
\end{center}
\caption{truc}
\label{shannon}
\end{figure}



\subsection{Comparaison entre catégories Free et Never}

\subsection{Sensibilité et Specificté du Pyo}

\section{Discussion}


\section{Conclusion}





\newpage



\newpage


\bibliographystyle{plain}
\bibliography{biblio}

\end{document}
